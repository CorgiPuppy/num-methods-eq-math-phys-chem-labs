\documentclass[12pt, a4paper]{report}
\usepackage[top=1cm, left=1cm, right=1cm]{geometry}

\usepackage[utf8]{inputenc}
\usepackage[russian]{babel}

\usepackage{array}
\newcolumntype{M}[1]{>{\centering\arraybackslash}m{#1}}

\usepackage{hyperref}
\hypersetup{
	colorlinks,
	citecolor=black,
	filecolor=black,
	linkcolor=black,
	urlcolor=black
}

\usepackage{sectsty}
\allsectionsfont{\centering}

\usepackage{indentfirst}
\setlength\parindent{24pt}

\usepackage{makecell}

\usepackage{amsmath}

\usepackage{tikz}
\usetikzlibrary{shapes.geometric, arrows}
\usetikzlibrary{arrows.meta}
\usetikzlibrary{decorations.text}

\begin{document}
	\begin{titlepage}
		\begin{center}
			\large \textbf{Министерство науки и высшего образования Российской Федерации} \\
			\large \textbf{Федеральное государственное бюджетное образовательное учреждение высшего образования} \\
			\large \textbf{«Российский химико-технологический университет имени Д.И. Менделеева»} \\

			\vspace*{4cm}
			\LARGE \textbf{ОТЧЕТ ПО ДОМАШНЕЙ РАБОТЕ №8}

			\vspace*{4cm}
			\begin{flushright}
				\Large
				\begin{tabular}{>{\raggedleft\arraybackslash}p{9cm} p{10cm}}
					Выполнил студент группы КС-36: & Золотухин Андрей Александрович \\
					Ссылка на репозиторий: & https://github.com/ \\
					& CorgiPuppy/ \\
					& num-methods-eq-math-phys-chem-labs \\
					Приняла: & Кольцова Элеонора Моисеевна \\
					Дата сдачи: & 14.04.2025 \\
				\end{tabular}
			\end{flushright}

			\vspace*{6cm}
			\Large \textbf{Москва \\ 2025}
		\end{center}
	\end{titlepage}

	\tableofcontents
	\thispagestyle{empty}
	\newpage

	\pagenumbering{arabic}

	\section*{Описание задачи}
	\addcontentsline{toc}{section}{Описание задачи}
	\large
	\begin{center}
		\begin{tabular}{||c|c|c||}
			\hline
			Уравнение & Интервалы переменных & Начальные и граничные условия \\

			\hline
			$ \frac{\partial u}{\partial t} = \frac{\partial^{2} u}{\partial x^{2}} + t^{2} + x^{2} + y^{2} $ & \makecell{$ x \in [0, 1] $ \\ $ t \in [0, 1] $} & \makecell{$ u(t = 0, x, y) = x^{2} + y^{2} $ \\ $\begin{cases} \frac{\partial u}{\partial x}(t, x = 0, y) = 0 \\ \frac{\partial u}{\partial x}(t, x = 1, y) = 2 \end{cases}$ \\ $\begin{cases} \frac{\partial u}{\partial y}(t, x, y = 0) = 8 \\ \frac{\partial u}{\partial y}(t, x, y = 1) = 2 \end{cases}$} \\

			\hline
		\end{tabular}
	\end{center}
	\par
	Для заданного уравнения:
	\begin{enumerate}
		\item записать явную разностную схему;
		\item проверить условие устойчивости разностной схемы;
		\item вывести рекуррентное соотношение;
		\item составить алгоритм (блок-схему) расчёта.
	\end{enumerate}

	\newpage

	\section*{Выполнение задачи}
	\addcontentsline{toc}{section}{Выполнение задачи}

	\subsection*{Задание 1}
	\addcontentsline{toc}{subsection}{Задание 1}
	\large
	Записать явную разностную схему:
\end{document}
