\documentclass[12pt, a4paper]{report}
\usepackage[top=1cm, left=1cm, right=1cm]{geometry}

\usepackage[utf8]{inputenc}
\usepackage[russian]{babel}

\usepackage{array}
\newcolumntype{M}[1]{>{\centering\arraybackslash}m{#1}}

\usepackage{hyperref}
\hypersetup{
	colorlinks,
	citecolor=black,
	filecolor=black,
	linkcolor=black,
	urlcolor=black
}

\usepackage{sectsty}
\allsectionsfont{\centering}

\usepackage{indentfirst}
\setlength\parindent{24pt}

\usepackage{makecell}

\usepackage{amsmath}

\usepackage{tikz}
\usetikzlibrary{shapes.geometric, arrows}
\usetikzlibrary{arrows.meta}
\usetikzlibrary{decorations.text}

\begin{document}
	\begin{titlepage}
		\begin{center}
			\large \textbf{Министерство науки и высшего образования Российской Федерации} \\
			\large \textbf{Федеральное государственное бюджетное образовательное учреждение высшего образования} \\
			\large \textbf{«Российский химико-технологический университет имени Д.И. Менделеева»} \\

			\vspace*{4cm}
			\LARGE \textbf{ОТЧЕТ ПО ДОМАШНЕЙ РАБОТЕ №8}

			\vspace*{4cm}
			\begin{flushright}
				\Large
				\begin{tabular}{>{\raggedleft\arraybackslash}p{9cm} p{10cm}}
					Выполнил студент группы КС-36: & Золотухин Андрей Александрович \\
					Ссылка на репозиторий: & https://github.com/ \\
					& CorgiPuppy/ \\
					& num-methods-eq-math-phys-chem-labs \\
					Приняла: & Кольцова Элеонора Моисеевна \\
					Дата сдачи: & 14.04.2025 \\
				\end{tabular}
			\end{flushright}

			\vspace*{6cm}
			\Large \textbf{Москва \\ 2025}
		\end{center}
	\end{titlepage}

	\tableofcontents
	\thispagestyle{empty}
	\newpage

	\pagenumbering{arabic}

	\section*{Описание задачи}
	\addcontentsline{toc}{section}{Описание задачи}
	\large
	\begin{center}
		\begin{tabular}{||c|c|c||}
			\hline
			Уравнение & Интервалы переменных & Начальные и граничные условия \\

			\hline
			$ \frac{\partial u}{\partial t} = \frac{\partial^{2} u}{\partial x^{2}} + t^{2} + x^{2} + y^{2} $ & \makecell{$ x \in [0, 1] $ \\ $ t \in [0, 1] $} & \makecell{$ u(t = 0, x, y) = x^{2} + y^{2} $ \\ $\begin{cases} \frac{\partial u}{\partial x}(t, x = 0, y) = 0 \\ \frac{\partial u}{\partial x}(t, x = 1, y) = 2 \end{cases}$ \\ $\begin{cases} \frac{\partial u}{\partial y}(t, x, y = 0) = 8 \\ \frac{\partial u}{\partial y}(t, x, y = 1) = 2 \end{cases}$} \\

			\hline
		\end{tabular}
	\end{center}
	\par
	Для заданного уравнения:
	\begin{enumerate}
		\item записать явную разностную схему;
		\item вывести рекуррентное соотношение;
		\item составить алгоритм (блок-схему) расчёта.
	\end{enumerate}

	\newpage

	\section*{Выполнение задачи}
	\addcontentsline{toc}{section}{Выполнение задачи}

	\subsection*{Задание 1}
	\addcontentsline{toc}{subsection}{Задание 1}
	\large
	Записать явную разностную схему:
	\small
	\begin{equation}\label{eq:explicit}
		\frac{u_{i, j}^{n+1} - u_{i, j}^{n+1}}{\Delta t} = \frac{u_{i+1, j}^{n} - 2u_{i, j}^{n} + u_{i-1, j}^{n}}{h_{x}^{2}} + \frac{u_{i, j+1}^{n} - 2u_{i, j}^{n} + u_{i, j-1}^{n}}{h_{y}^{2}} + (n \Delta t)^{2} + ((i - 1)h_{x})^{2} + ((j - 1)h_{y})^{2}.
	\end{equation}

	\subsection*{Задание 2}
	\addcontentsline{toc}{subsection}{Задание 2}
	\large
	Вывести рекуррентное соотношение: \par
	Выражаю из раностной схемы \eqref{eq:explicit} величину $u_{i, j}^{n+1}$:
	\begin{equation}\label{eq:recurrence}
		\small
		u_{i, j}^{n+1} = u_{i, j}^{n} + \frac{\Delta t}{h_{x}^{2}}(u_{i+1, j}^{n} - 2u_{i, j}^{n} + u_{i-1, j}^{n}) + \frac{\Delta t}{h_{y}^{2}}(u_{i, j+1}^{n} - 2u_{i, j}^{n} + u_{i, j-1}^{n}) + \Delta t((n \Delta t)^{2} + ((i - 1)h_{x})^{2} + ((j - 1)h_{y})^{2}). 
	\end{equation}
	\par
	Соотношение типа \eqref{eq:recurrence}, позволяющее рассчитывать значения искомой функции в узлах разностной сетки через известные значения в других узлах разностной сетки, называют \textbf{рекуррентным соотношением}.

	\subsection*{Задание 3}
	\addcontentsline{toc}{subsection}{Задание 3}
	\large
	Составить алгоритм (блок-схему) расчёта: \par
	\small
	\tikzstyle{start} = [circle, draw=black!60, fill=white!5, very thick, minimum size=10mm]
	\tikzstyle{stop} = [circle, color=black!60, fill=black!60, very thick, minimum size=10mm]
	\tikzstyle{process} = [rectangle, minimum width=3cm, minimum height=1cm, text centered, draw=black]
	\tikzstyle{decision} = [diamond, minimum width=3cm, minimum height=1cm, text centered, draw=black]
	\tikzstyle{arrow} = [thick,->,>=stealth]
	\begin{center}
		\begin{tikzpicture}[node distance=2cm]
			\node (start) [start] {};
			\node (in1) [process, below of=start] {\makecell{Задание начальных условий: \\ цикл по $i=1..N_{x}$; цикл по $j=1..N_{x}$ \\ $u_{i, j}^{0} = ((i - 1)h_{x})^{2} + ((j - 1)h_{y})^{2}$}};
			\node (in2) [process, below of=in1] {$n = 0$};
			\node (dec) [decision, below of=in2, yshift=-1cm] {$n = N_{t}$};
			\node (stop) [stop, right of=dec, xshift=2cm] {};
			\node (pr1) [process, below of=dec, yshift=-1cm] {\makecell{цикл по $i=2..N_{x}-1$; цикл по $j=2..N_{y}-1$ \\ $u_{i, j}^{n+1} = u_{i, j}^{n} + \frac{\Delta t}{h_{x}^{2}}(u_{i+1, j}^{n} - 2u_{i, j}^{n} + u_{i-1, j}^{n}) + \frac{\Delta t}{h_{y}^{2}}(u_{i, j+1}^{n} - 2u_{i, j}^{n} + u_{i, j-1}^{n}) + \Delta t((n \Delta t)^{2} + ((i - 1)h_{x})^{2} + ((j - 1)h_{y})^{2})$ }};
			\node (pr2) [process, below of=pr1, yshift=-1cm] {\makecell{Определение $u_{1, j}^{n+1}$, $u_{N_{x}, j}^{n+1}$, $u_{i, 1}^{n+1}$ и $u_{i, N_{y}}^{n+1}$ из граничных условий: \\ цикл по $j=1..N_{y}$: $\begin{cases} u_{1, j}^{n+1} = 0 \\ u_{N_{x}, j} = 2 \end{cases}$ \\ цикл по $i=1..N_{x}$: $\begin{cases} u_{j, 1}^{n+1} = 8 \\ u_{j, N_{y}} = 2 \end{cases}$}};
			\node (pr3) [process, below of=pr2, yshift=-1cm] {$n = n + 1$};
			\draw [arrow] (start) -- (in1);
			\draw [arrow] (in1) -- (in2);
			\draw [arrow] (in2) -- (dec);
			\draw [arrow] (dec) -- node[anchor=south] {да} (stop);
			\draw [arrow] (dec) -- node[anchor=west] {нет} (pr1);
			\draw [arrow] (pr1) -- (pr2);
			\draw [arrow] (pr2) -- (pr3);
			\draw [arrow] (pr3.west) -- ++(left:8.2cm) -| +(up:9cm) -- (dec.west);
		\end{tikzpicture}
	\end{center}
\end{document}
