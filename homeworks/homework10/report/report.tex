\documentclass[12pt, a4paper]{report}
\usepackage[top=1cm, left=1cm, right=1cm]{geometry}

\usepackage[utf8]{inputenc}
\usepackage[russian]{babel}

\usepackage{array}
\newcolumntype{M}[1]{>{\centering\arraybackslash}m{#1}}

\usepackage{hyperref}
\hypersetup{
	colorlinks,
	citecolor=black,
	filecolor=black,
	linkcolor=black,
	urlcolor=black
}

\usepackage{sectsty}
\allsectionsfont{\centering}

\usepackage{indentfirst}
\setlength\parindent{24pt}

\usepackage{makecell}

\usepackage{amsmath}

\def\H{\rule{0pt}{1.5ex}H}

\usepackage{tikz}
\usetikzlibrary{shapes.geometric, arrows}
\usetikzlibrary{arrows.meta}
\usetikzlibrary{decorations.text}

\begin{document}
	\begin{titlepage}
		\begin{center}
			\large \textbf{Министерство науки и высшего образования Российской Федерации} \\
			\large \textbf{Федеральное государственное бюджетное образовательное учреждение высшего образования} \\
			\large \textbf{«Российский химико-технологический университет имени Д.И. Менделеева»} \\

			\vspace*{4cm}
			\LARGE \textbf{ОТЧЕТ ПО ДОМАШНЕЙ РАБОТЕ №10}

			\vspace*{4cm}
			\begin{flushright}
				\Large
				\begin{tabular}{>{\raggedleft\arraybackslash}p{9cm} p{10cm}}
					Выполнил студент группы КС-36: & Золотухин Андрей Александрович \\
					Ссылка на репозиторий: & https://github.com/ \\
					& CorgiPuppy/ \\
					& num-methods-eq-math-phys-chem-labs \\
					Приняла: & Кольцова Элеонора Моисеевна \\
					Дата сдачи: & 28.04.2025 \\
				\end{tabular}
			\end{flushright}

			\vspace*{6cm}
			\Large \textbf{Москва \\ 2025}
		\end{center}
	\end{titlepage}

	\tableofcontents
	\thispagestyle{empty}
	\newpage

	\pagenumbering{arabic}

	\section*{Описание задачи}
	\addcontentsline{toc}{section}{Описание задачи}
	\large
	\begin{center}
		\begin{tabular}{||c|c|c||}
			\hline
			Уравнение & \makecell{Интервалы \\ переменных} & Начальные и граничные условия \\

			\hline
			$ \frac{\partial u}{\partial t} = 16\frac{\partial^{2} u}{\partial x^{2}} + 12\frac{\partial^{2} u}{\partial y^{2}} + txe^{y} $ & \makecell{$ x \in [0, 1] $ \\ $ y \in [0, 1] $ \\ $ t \in [0, 1] $} & \makecell{$ u(t = 0, x, y) = 0 $ \\ $\begin{cases} \frac{\partial u}{\partial x}(t, x = 0, y) = te^{y} \\ u(t, x = 1, y) = te^{y} \end{cases}$ \\ $\begin{cases} u(t, x, y = 0) = tx \\ \frac{\partial u}{\partial y}(t, x, y = 1) = 2.7tx \end{cases}$} \\

			\hline
		\end{tabular}
	\end{center}
	\par
	Для заданного уравнения:
	\begin{enumerate}
		\item записать неявную разностную схему;
		\item получить условие устойчивости разностной схемы;
		\begin{enumerate}
			\item записать схему расщепления;
			\item определить порядок аппроксимации разностной схемы;
			\item привести схемы к виду, удобному для использования метода прогонки;
			\item проверить сходимость прогонки;
			\item записать рекуррентное прогоночное соотношение;
			\item составить алгоритм (блок-схему) расчёта.
		\end{enumerate}
		\begin{enumerate}
			\item записать схему переменных направлений;
			\item определить порядок аппроксимации разностной схемы;
			\item привести схемы к виду, удобному для использования метода прогонки;
			\item проверить сходимость прогонки;
			\item записать рекуррентное прогоночное соотношение;
			\item составить алгоритм (блок-схему) расчёта.
		\end{enumerate}
		\begin{enumerate}
			\item записать схему предиктор-корректор;
			\item определить порядок аппроксимации разностной схемы;
			\item привести схемы к виду, удобному для использования метода прогонки;
			\item проверить сходимость прогонки;
			\item записать рекуррентное прогоночное соотношение для предиктора;
			\item записать рекуррентное прогоночное соотношение для корректора;
			\item составить алгоритм (блок-схему) расчёта.
		\end{enumerate}
	\end{enumerate}

	\newpage

	\section*{Выполнение задачи}
	\addcontentsline{toc}{section}{Выполнение задачи}

	\subsection*{Задание 1}
	\addcontentsline{toc}{subsection}{Задание 1}
	\large
	Записать неявную разностную схему:
	\small
	\begin{equation}\label{eq:implicit}
		\frac{u_{i, j}^{n+1} - u_{i, j}^{n}}{\Delta t} = 16\frac{u_{i+1, j}^{n+1} - 2u_{i, j}^{n+1} + u_{i-1, j}^{n+1}}{h_{x}^{2}} + 12\frac{u_{i, j+1}^{n+1} - 2u_{i, j}^{n+1} + u_{i, j-1}^{n+1}}{h_{y}^{2}} + (n+1)\Delta t(i-1)h_{x}e^{y}.
	\end{equation}

	\subsection*{Задание 2}
	\addcontentsline{toc}{subsection}{Задание 2}
	\large
	Получить условие устойчивости разностной схемы: \par
	Исследую устойчивость неявной разностной схемы \eqref{eq:implicit}, аппроксимирующей исходное дифференциальное уравнение, с помощью спектрального метода. Для этого отброшу член $(n+1)\Delta t(i-1)h_{x}e^{y}$, наличие которого не оказывает влияния на устойчивость разностной схемы, и представлю решение в виде гармоники:
	\begin{equation}\label{eq:harmonic}
		u_{i, j}^{n} = \lambda^{n}e^{i \alpha i}e^{i \alpha j}, \> \alpha \in [0, 2\pi], \beta \in [0, 2\pi].
	\end{equation}
	\par
	Подставляя \eqref{eq:harmonic} в разностную схему \eqref{eq:implicit}, получаю:
	\scriptsize
	\begin{equation*}
		\frac{\lambda^{n+1}e^{i \alpha i}e^{i \alpha j} - \lambda^{n}e^{i \alpha i}e^{i \alpha j}}{\Delta t} = 16\frac{\lambda^{n+1}e^{i \alpha (i+1)}e^{i \alpha j} - 2\lambda^{n+1}e^{i \alpha i}e^{i \alpha j} + \lambda^{n+1}e^{i \alpha (i-1)}e^{i \alpha j}}{h_{x}^{2}} + 12\frac{\lambda^{n+1}e^{i \beta i}e^{i \beta (j+1)} - 2\lambda^{n+1}e^{i \beta i}e^{i \beta j} + \lambda^{n+1}e^{i \beta i)}e^{i \beta (j-1)}}{h_{y}^{2}}.
	\end{equation*}
	\large
	\par
	Упрощая полученное выражение, деля левую и правую его части на $\lambda^{n}e^{i \alpha i}e^{i \alpha j}$:
	\begin{equation*}
		\frac{\lambda - 1}{\Delta t} = 16\lambda\frac{e^{i \alpha} - 2 + e^{-i \alpha}}{h_{x}^{2}} + 12\lambda\frac{e^{i \beta} - 2 + e^{-i \beta}}{h_{y}^{2}}.
	\end{equation*}
	\par
	Преобразую комплексные числа из экспоненциальной формы в тригонометрическую:
	\begin{equation*}
		e^{\pm i \alpha} = \cos{\alpha} \pm i\sin{\alpha}, e^{\pm i \beta} = \cos{\beta} \pm i\sin{\beta} \Rightarrow \frac{\lambda - 1}{\Delta t} = 16\lambda\frac{2\cos{\alpha}-2}{h_{x}^{2}} + 12\lambda\frac{2\cos{\beta}-2}{h_{y}^{2}}.
	\end{equation*}
	\par
	Используя тригонометрические тождества
	\begin{equation*}
		\cos{\alpha} = \cos^{2}{\frac{\alpha}{2}} - \sin^{2}{\frac{\alpha}{2}} = 1 - 2\sin^{2}{\frac{\alpha}{2}}, \cos{\beta} = \cos^{2}{\frac{\beta}{2}} - \sin^{2}{\frac{\beta}{2}} = 1 - 2\sin^{2}{\frac{\beta}{2}},
	\end{equation*}
	\par
	получаю формулу, из которой затем выражаю $\lambda$:
	\begin{equation*}
		\frac{\lambda - 1}{\Delta t} = \frac{-64\lambda\sin^{2}{\frac{\alpha}{2}}}{h_{x}^{2}} + \frac{-48\lambda\sin^{2}{\frac{\beta}{2}}}{h_{y}^{2}} \Rightarrow \lambda = \frac{1}{1 + \frac{64\Delta t\sin^{2}{\frac{\alpha}{2}}}{h_{x}^{2}} + \frac{48\Delta t\sin^{2}{\frac{\beta}{2}}}{h_{y}^{2}}}.
	\end{equation*}
	\par
	С учётом необходимого условия устойчивости разностных схем $\lvert \lambda \rvert \le 1$ имею:
	\begin{equation*}
		-1 \le \frac{1}{1 + \frac{64\Delta t\sin^{2}{\frac{\alpha}{2}}}{h_{x}^{2}} + \frac{48\Delta t\sin^{2}{\frac{\beta}{2}}}{h_{y}^{2}}} \le 1.
	\end{equation*}
	\par
	В полученном двойном неравенстве левое и правое условие выполняются автоматически. \par
	Для любых значений $\Delta t$, $h_{x}$ и $h_{y}$ неравенство выполняется. Следовательно, разностная схема \textbf{абсолютно устойчива}.

	\subsection*{Задание 3}
	\addcontentsline{toc}{subsection}{Задание 3}
	\large
	Записать схему расщепления: \par
	Рассмотрю метод разрешения неявной разностной схемы \eqref{eq:implicit}, называемый \textbf{методом дробных шагов}. Данный метод позволяет представить разностной схему \eqref{eq:implicit} в виде двух подсхем, каждая из которых может быть решена с помощью метода прогонки. \par
	Разобью пополам интервал $\Delta t$ между точками $t^{n}$ и $t^{n+1}$ на разностной сетке и обозначу полученную промежуточную точку как $t^{n+\frac{1}{2}}$. \par
	Запишу на первом полушаге интервала $\Delta t$ неявную разностную схему, которая будет учитывать только производную второго порядка по координате \textit{x}:
	\small
	\begin{equation}\label{eq:firstSplittingSubscheme}
		\frac{u_{i, j}^{n+\frac{1}{2}} - u_{i, j}^{n}}{\Delta t} = 16\frac{u_{i+1, j}^{n+\frac{1}{2}} - 2u_{i, j}^{n+\frac{1}{2}} + u_{i-1, j}^{n+\frac{1}{2}}}{h_{x}^{2}} + (n + \frac{1}{2})\Delta t(i-1)h_{x}e^{(j-1)h_{y}}.
	\end{equation}
	\par
	\large
	Запишу на втором полушаге интервала $\Delta t$ неявную разностную схему, которая будет учитывать только производную вторую порядка по координате \textit{y}:
	\begin{equation}\label{eq:secondSplittingSubscheme}
		\frac{u_{i, j}^{n+1} - u_{i, j}^{n+\frac{1}{2}}}{\Delta t} = 12\frac{u_{i, j+1}^{n+1} - 2u_{i, j}^{n+1} + u_{i, j-1}^{n+1}}{h_{y}^{2}}.
	\end{equation}
	\par
	Складывая подсхемы \eqref{eq:firstSplittingSubscheme} и \eqref{eq:secondSplittingSubscheme}, получаю соотношение, отличающееся от неявной разностной схемы \eqref{eq:implicit} только тем, что вторая производная по координате \textit{x} аппроксимирована в нём не на \textit{(n+1)}-м шаге по времени, а на шаге ($n+\frac{1}{2}$):
	\small
	\begin{equation}\label{eq:splittingScheme}
		\frac{u_{i, j}^{n+1} - u_{i, j}^{n}}{\Delta t} = 16\frac{u_{i+1, j}^{n+\frac{1}{2}} - 2u_{i, j}^{n+\frac{1}{2}} + u_{i-1, j}^{n+\frac{1}{2}}}{h_{x}^{2}} + 12\frac{u_{i, j+1}^{n+1} - 2u_{i, j}^{n+1} + u_{i, j-1}^{n+1}}{h_{y}^{2}} + (n + \frac{1}{2})\Delta t(i-1)h_{x}e^{(j-1)h_{y}}.
	\end{equation}
	\par
	\large
	Таким образом, дифференциальное уравнение из условия задачи может быть аппроксимировано с помощью последовательного разрешения двух подсхем \eqref{eq:firstSplittingSubscheme}, \eqref{eq:secondSplittingSubscheme}, называемых в совокупности \textbf{схемой расщепления}.

	\subsection*{Задание 4}
	\addcontentsline{toc}{subsection}{Задание 4}
	\large
	Определить порядок аппроксимации разностной схемы расщепления \eqref{eq:splittingScheme}:

	\subsection*{Задание 5}
	\addcontentsline{toc}{subsection}{Задание 5}
	\large
	Привести схемы к виду, удобному для использования метода прогонки: \par
	\subsubsection*{Первая подсхема}
	\large
	Приведу подсхему \eqref{eq:firstSplittingSubscheme} к виду, удобному для использования метода прогонки:
	\small
	\begin{equation*}
		-16\frac{\Delta t}{h_{x}^{2}}u_{i+1, j}^{n+\frac{1}{2}} + (1 + 32\frac{\Delta t}{h_{x}^{2}})u_{i, j}^{n+\frac{1}{2}} - 16\frac{\Delta t}{h_{x}^{2}}u_{i-1, j}^{n+\frac{1}{2}} = u_{i, j}^{n} + (n + \frac{1}{2})\Delta t(i-1)h_{x}e^{(j-1)h_{y}}.
	\end{equation*}
	\subsubsection*{Вторая подсхема}
	\large
	Приведу подсхему \eqref{eq:secondSplittingSubscheme} к виду, удобному для использования метода прогонки:
	\begin{equation*}
		-12\frac{\Delta t}{h_{y}^{2}}u_{i, j+1}^{n+1} + (1 + 24\frac{\Delta t}{h_{y}^{2}})u_{i, j}^{n+1} - 12\frac{\Delta t}{h_{y}^{2}}u_{i, j-1}^{n+1} = u_{i, j}^{n+\frac{1}{2}}.
	\end{equation*}

	\subsection*{Задание 6}
	\addcontentsline{toc}{subsection}{Задание 6}
	\large
	Проверить сходимость прогонки:
	\subsubsection*{Первая подсхема}
	\large
	Коэффициенты, соответствующие уравнению \eqref{eq:firstSplittingSubscheme}, имеют вид:
	\small
	\begin{center}
		$a_{i}=c_{i}=-16\frac{\Delta t}{h_{x}^{2}}$, $\>$ $b_{i}=1 + 32\frac{\Delta t}{h_{x}^{2}}$, $\>$ $\xi_{i, j}^{n}=u_{i, j}^{n} + (n + \frac{1}{2})\Delta t(i-1)h_{x}e^{(j-1)h_{y}}$.
	\end{center}
	\par
	\large
	Легко видеть, что для первой подсхемы \eqref{eq:firstSplittingSubscheme} схемы расщепления достаточное условие сходимости прогонки выполняется:
	\begin{equation*}
		\lvert a_{i} \rvert + \lvert c_{i} \rvert = 32\frac{\Delta t}{h_{x}^{2}} < 1 + 32\frac{\Delta t}{h_{x}^{2}} = \lvert b_{i} \rvert.
	\end{equation*}
	\subsubsection*{Вторая подсхема}
	\large
	Коэффициенты, соответствующие уравнению \eqref{eq:secondSplittingSubscheme}, имеют вид:
	\small
	\begin{center}
		$\tilde{a}_{j}=\tilde{c}_{j}=-12\frac{\Delta t}{h_{y}^{2}}$, $\>$ $\tilde{b}_{j}=1 + 24\frac{\Delta t}{h_{y}^{2}}$, $\>$ $\tilde{\xi}_{i, j}^{n}=u_{i, j}^{n} + (n + \frac{1}{2})\Delta t(i-1)h_{x}e^{(j-1)h_{y}}$.
	\end{center}
	\par
	\large
	Легко видеть, что для второй подсхемы \eqref{eq:secondSplittingSubscheme} схемы расщепления достаточное условие сходимости прогонки выполняется:
	\begin{equation*}
		\lvert \tilde{a}_{j} \rvert + \lvert \tilde{c}_{j} \rvert = 24\frac{\Delta t}{h_{y}^{2}} < 1 + 24\frac{\Delta t}{h_{y}^{2}} = \lvert \tilde{b}_{j} \rvert.
	\end{equation*}

	\subsection*{Задание 7}
	\addcontentsline{toc}{subsection}{Задание 7}
	\large
	Записать рекуррентное прогоночное соотношение:
	\subsubsection*{Первая подсхема}
	\large
	Рекуррентное прогоночное соотношение для первой подсхемы \eqref{eq:firstSplittingSubscheme} имеет вид:
	\begin{equation*}
		u_{i, j}^{n+\frac{1}{2}} = \alpha_{i}u_{i+1, j}^{n+\frac{1}{2}} + \beta_{i}.
	\end{equation*}
	\par
	Прогоночные коэффициенты:
	\begin{equation*}
		\alpha_{i} = -\frac{a_{i}}{b_{i} + c_{i}\alpha_{i-1}}, \> \beta_{i} = \frac{\xi_{i, j}^{n} - c_{i}\beta_{i-1}}{b_{i} + c_{i}\alpha_{i-1}}.
	\end{equation*}
	\subsubsection*{Вторая подсхема}
	\large
	Рекуррентное прогоночное соотношение для второй подсхемы \eqref{eq:secondSplittingSubscheme} имеет вид:
	\begin{equation*}
		u_{i, j}^{n+1} = \tilde{\alpha}_{j}u_{i, j+1}^{n+1} + \tilde{\beta}_{i}.
	\end{equation*}
	\par
	Прогоночные коэффициенты:
	\begin{equation*}
		\tilde{\alpha}_{j} = -\frac{\tilde{a}_{j}}{\tilde{b}_{j} + \tilde{c}_{j}\tilde{\alpha}_{j-1}}, \> \tilde{\beta}_{j} = \frac{\tilde{\xi}_{i, j}^{n+\frac{1}{2}} - \tilde{c}_{j}\tilde{\beta}_{j-1}}{\tilde{b}_{j} + \tilde{c}_{j}\tilde{\alpha}_{j-1}}.
	\end{equation*}

	\subsection*{Задание 8}
	\addcontentsline{toc}{subsection}{Задание 8}
	\large
	Составить алгоритм (блок-схему) расчёта:
	\small
	\tikzstyle{start} = [circle, draw=black!60, fill=white!5, very thick, minimum size=10mm]
	\tikzstyle{stop} = [circle, color=black!60, fill=black!60, very thick, minimum size=10mm]
	\tikzstyle{process} = [rectangle, minimum width=3cm, minimum height=1cm, text centered, draw=black]
	\tikzstyle{decision} = [diamond, minimum width=3cm, minimum height=1cm, text centered, draw=black]
	\tikzstyle{arrow} = [thick,->,>=stealth]
	\begin{center}
		\begin{tikzpicture}[node distance=2cm]
			\node (start) [start] {};
			\node (in1) [process, below of=start] {
				\makecell{
					Задание начальных условий: \\ цикл по $i=1..N_{x}$; цикл по $j=1..N_{x}$ \\ 
					$u_{i, j}^{0} = 0$
				}
			};
			\node (in2) [process, below of=in1] {
				$n = 0$
			};
			\node (dec) [decision, below of=in2, yshift=-1cm] {
				$n = N_{t}$
			};
			\node (stop) [stop, right of=dec, xshift=2cm] {};
			\node (pr1) [process, below of=dec, yshift=-1cm] {
				\makecell{
					цикл по $j=2..N_{y}-1$ \\
					1) $\alpha_{1} = 1$, $\beta_{1} = -((n+\frac{1}{2})\Delta te^{(j-1)h_{y}})h_{x}$; \\
					2) цикл по $i=2..N_{x}-1$: расчёт $a_{i}$, $b_{i}$, $c_{i}$, $\xi_{i, j}^{n}$; $\alpha_{i}$, $\beta_{i}$; \\
					3) $u_{N_{x}, j}^{n+\frac{1}{2}} = (n+\frac{1}{2})\Delta te^{(j-1)h_{y}}$; \\
					4) цикл по $i=N_{x}-1..1$: $u_{i, j}^{n+\frac{1}{2}} = \alpha_{i}u_{i+1, j}^{n+\frac{1}{2}} + \beta_{i}$.
				}
			};
			\node (pr2) [process, below of=pr1, yshift=-2cm] {
				\makecell{
					цикл по $i=2..N_{x}-1$ \\
					1) $\tilde{\alpha}_{1} = 0$, $\tilde{\beta}_{1} = (n+1)\Delta t(i-1)h_{x}$; \\
					2) цикл по $j=2..N_{y}-1$: расчёт $\tilde{a}_{j}$, $\tilde{b}_{j}$, $\tilde{c}_{j}$, $\tilde{\xi}_{i, j}^{n+\frac{1}{2}}$; $\tilde{\alpha}_{j}$, $\tilde{\beta}_{j}$; \\
					3) $u_{i, N_{y}}^{n+1} = u_{i, N_{y}-1}^{n+1} + (2.7(n+1)\Delta t)h_{y}$; \\
					4) цикл по $j=N_{y}-1..1$: $u_{i, j}^{n+1} = \tilde{\alpha}_{j}u_{i, j+1}^{n+1} + \tilde{\beta}_{j}$.
				}
			};
			\node (pr3) [process, below of=pr2, yshift=-1cm] {
				\makecell{
					цикл по $j=1..N_{y}$ \\
					$u_{1, j}^{n+1} = u_{2, j}^{n+1} + ((n+1)\Delta te^{(j-1)h_{y}})h_{x}$ \\
					$u_{N_{x}, j}^{n+1} = (n+1)\Delta te^{(j-1)h_{y}}$
				}
			};
			\node (pr4) [process, below of=pr3, yshift=-1cm] {
				\makecell{
					цикл по $i=1..N_{x}$ \\
					$u_{i, 1}^{n+1} = (n+1)\Delta t(i-1)h_{x}$ \\
					$u_{i, N_{y}}^{n+1} = u_{i, N_{y}}^{n+1} + ((n+1)\Delta t(i-1)h_{x})h_{y}$
				}
			};
			\node (pr5) [process, below of=pr4, yshift=-1cm] {$n = n + 1$};
			\draw [arrow] (start) -- (in1);
			\draw [arrow] (in1) -- (in2);
			\draw [arrow] (in2) -- (dec);
			\draw [arrow] (dec) -- node[anchor=south] {да} (stop);
			\draw [arrow] (dec) -- node[anchor=west] {нет} (pr1);
			\draw [arrow] (pr1) -- (pr2);
			\draw [arrow] (pr2) -- (pr3);
			\draw [arrow] (pr3) -- (pr4);
			\draw [arrow] (pr4) -- (pr5);
			\draw [arrow] (pr5.west) -- ++(left:8.2cm) -| +(up:16cm) -- (dec.west);
		\end{tikzpicture}
	\end{center}

	\subsection*{Задание 9}
	\addcontentsline{toc}{subsection}{Задание 9}
	\large
	Записать схему переменных направлений:
	\begin{equation}\label{eq:variableDirectionsScheme}
		\makecell{
			\frac{u_{i, j}^{n+\frac{1}{2}} - u_{i, j}^{n}}{\Delta t} = \frac{16}{2}\frac{u_{i+1, j}^{n+\frac{1}{2}} - 2u_{i, j}^{n+\frac{1}{2}} + u_{i-1, j}^{n+\frac{1}{2}}}{h_{x}^{2}} + \frac{12}{2}\frac{u_{i, j+1}^{n} - 2u_{i, j}^{n} + u_{i, j-1}^{n}}{h_{y}^{2}}, \\ 
			\frac{u_{i, j}^{n+1} - u_{i, j}^{n+\frac{1}{2}}}{\Delta t} = \frac{16}{2}\frac{u_{i+1, j}^{n+\frac{1}{2}} - 2u_{i, j}^{n+\frac{1}{2}} + u_{i-1, j}^{n+\frac{1}{2}}}{h_{x}^{2}} + \frac{12}{2}\frac{u_{i, j+1}^{n+1} - 2u_{i, j}^{n+1} + u_{i, j-1}^{n+1}}{h_{y}^{2}} + (n+\frac{1}{2})\Delta t(i-1)h_{x}e^{(j-1)h_{y}}.
		}
	\end{equation}
	\par
	Первая подсхема в схеме переменных направлений \eqref{eq:variableDirectionsScheme} аппроксимирует производную по времени на первом полушаге интервала $\Delta t$ и является неявной по координате \textit{x} и явной по координате \textit{y}. Вторая подсхема аппроксимирует производную по времени на втором полушаге интервала $\Delta t$ и является неявной по координате \textit{y} и явной по координате \textit{x}. Каждая из подсхем (как и в случае \textit{схемы расщепления} \eqref{eq:firstSplittingSubscheme}, \eqref{eq:secondSplittingSubscheme}) является \textbf{абсолютно устойчивой}.

	\subsection*{Задание 10}
	\addcontentsline{toc}{subsection}{Задание 10}
	\large
	Определить порядок аппроксимации разностной схемы:

	\subsection*{Задание 11}
	\addcontentsline{toc}{subsection}{Задание 11}
	\large
	Привести схемы к виду, удобному для использования метода прогонки: \par
	\subsubsection*{Первая подсхема}
	\large
	Приведу первую подсхему \eqref{eq:variableDirectionsScheme} к виду, удобному для использования метода прогонки:
	\small
	\begin{equation*}
		-\frac{16}{2}\frac{\Delta t}{h_{x}^{2}}u_{i+1, j}^{n+\frac{1}{2}} + (1 + 16\frac{\Delta t}{h_{x}^{2}})u_{i, j}^{n+\frac{1}{2}} - \frac{16}{2}\frac{\Delta t}{h_{x}^{2}}u_{i-1, j}^{n+\frac{1}{2}} = u_{i, j}^{n} + \frac{12}{2}\Delta t\frac{u_{i, j+1}^{n} - 2u_{i, j}^{n} + u_{i, j-1}^{n}}{h_{y}^{2}}.
	\end{equation*}
	\subsubsection*{Вторая подсхема}
	\large
	Приведу вторую подсхему \eqref{eq:variableDirectionsScheme} к виду, удобному для использования метода прогонки:
	\small
	\begin{equation*}
		-\frac{12}{2}\frac{\Delta t}{h_{y}^{2}}u_{i, j+1}^{n+1} + (1 + 12\frac{\Delta t}{h_{y}^{2}})u_{i, j}^{n+1} - \frac{12}{2}\frac{\Delta t}{h_{y}^{2}}u_{i, j-1}^{n+1} = u_{i, j}^{n+\frac{1}{2}} + \frac{16}{2}\frac{u_{i+1, j}^{n+\frac{1}{2}} - 2u_{i, j}^{n+\frac{1}{2}} + u_{i-1, j}^{n+\frac{1}{2}}}{h_{x}^{2}} + \Delta t((n+\frac{1}{2})\Delta t(i-1)h_{x}e^{(j-1)h_{y}}).
	\end{equation*}

	\subsection*{Задание 12}
	\addcontentsline{toc}{subsection}{Задание 12}
	\large
	Проверить сходимость прогонки:
	\subsubsection*{Первая подсхема}
	\large
	Коэффициенты, соответствующие уравнению первой подсхемы \eqref{eq:variableDirectionsScheme}, имеют вид:
	\small
	\begin{center}
		$a_{i}=c_{i}=-\frac{16}{2}\frac{\Delta t}{h_{x}^{2}}$, $\>$ $b_{i}=1 + 16\frac{\Delta t}{h_{x}^{2}}$, $\>$ $\xi_{i, j}^{n}=u_{i, j}^{n} + \frac{12}{2}\Delta t\frac{u_{i, j+1}^{n} - 2u_{i, j}^{n} + u_{i, j-1}^{n}}{h_{y}^{2}}$.
	\end{center}
	\par
	\large
	Легко видеть, что для первой подсхемы \eqref{eq:variableDirectionsScheme} схемы расщепления достаточное условие сходимости прогонки выполняется:
	\begin{equation*}
		\lvert a_{i} \rvert + \lvert c_{i} \rvert = 16\frac{\Delta t}{h_{x}^{2}} < 1 + 16\frac{\Delta t}{h_{x}^{2}} = \lvert b_{i} \rvert.
	\end{equation*}
	\subsubsection*{Вторая подсхема}
	\large
	Коэффициенты, соответствующие уравнению второй подсхемы \eqref{eq:variableDirectionsScheme}, имеют вид:
	\small
	\begin{center}
		$\tilde{a}_{j}=\tilde{c}_{j}=-\frac{12}{2}\frac{\Delta t}{h_{y}^{2}}$, $\>$ $\tilde{b}_{j}=1 + 12\frac{\Delta t}{h_{y}^{2}}$, $\>$ $\tilde{\xi}_{i, j}^{n}=u_{i, j}^{n+\frac{1}{2}} + \frac{16}{2}\frac{u_{i+1, j}^{n+\frac{1}{2}} - 2u_{i, j}^{n+\frac{1}{2}} + u_{i-1, j}^{n+\frac{1}{2}}}{h_{x}^{2}} + \Delta t((n+\frac{1}{2})\Delta t(i-1)h_{x}e^{(j-1)h_{y}})$.
	\end{center}
	\par
	\large
	Легко видеть, что для второй подсхемы \eqref{eq:variableDirectionsScheme} схемы расщепления достаточное условие сходимости прогонки выполняется:
	\begin{equation*}
		\lvert \tilde{a}_{j} \rvert + \lvert \tilde{c}_{j} \rvert = 12\frac{\Delta t}{h_{y}^{2}} < 1 + 12\frac{\Delta t}{h_{y}^{2}} = \lvert \tilde{b}_{j} \rvert.
	\end{equation*}

	\subsection*{Задание 13}
	\addcontentsline{toc}{subsection}{Задание 13}
	\large
	Записать рекуррентное прогоночное соотношение:
	\subsubsection*{Первая подсхема}
	\large
	Рекуррентное прогоночное соотношение для первой подсхемы \eqref{eq:variableDirectionsScheme} имеет вид:
	\begin{equation*}
		u_{i, j}^{n+\frac{1}{2}} = \alpha_{i}u_{i+1, j}^{n+\frac{1}{2}} + \beta_{i}.
	\end{equation*}
	\par
	Прогоночные коэффициенты:
	\begin{equation*}
		\alpha_{i} = -\frac{a_{i}}{b_{i} + c_{i}\alpha_{i-1}}, \> \beta_{i} = \frac{\xi_{i, j}^{n} - c_{i}\beta_{i-1}}{b_{i} + c_{i}\alpha_{i-1}}.
	\end{equation*}
	\subsubsection*{Вторая подсхема}
	\large
	Рекуррентное прогоночное соотношение для второй подсхемы \eqref{eq:variableDirectionsScheme} имеет вид:
	\begin{equation*}
		u_{i, j}^{n+1} = \tilde{\alpha}_{j}u_{i, j+1}^{n+1} + \tilde{\beta}_{i}.
	\end{equation*}
	\par
	Прогоночные коэффициенты:
	\begin{equation*}
		\tilde{\alpha}_{j} = -\frac{\tilde{a}_{j}}{\tilde{b}_{j} + \tilde{c}_{j}\tilde{\alpha}_{j-1}}, \> \tilde{\beta}_{j} = \frac{\tilde{\xi}_{i, j}^{n+\frac{1}{2}} - \tilde{c}_{j}\tilde{\beta}_{j-1}}{\tilde{b}_{j} + \tilde{c}_{j}\tilde{\alpha}_{j-1}}.
	\end{equation*}

	\subsection*{Задание 14}
	\addcontentsline{toc}{subsection}{Задание 14}
	\large
	Составить алгоритм (блок-схему) расчёта:
	\small
	\tikzstyle{start} = [circle, draw=black!60, fill=white!5, very thick, minimum size=10mm]
	\tikzstyle{stop} = [circle, color=black!60, fill=black!60, very thick, minimum size=10mm]
	\tikzstyle{process} = [rectangle, minimum width=3cm, minimum height=1cm, text centered, draw=black]
	\tikzstyle{decision} = [diamond, minimum width=3cm, minimum height=1cm, text centered, draw=black]
	\tikzstyle{arrow} = [thick,->,>=stealth]
	\begin{center}
		\begin{tikzpicture}[node distance=2cm]
			\node (start) [start] {};
			\node (in1) [process, below of=start] {
				\makecell{
					Задание начальных условий: \\ цикл по $i=1..N_{x}$; цикл по $j=1..N_{x}$ \\ 
					$u_{i, j}^{0} = 0$
				}
			};
			\node (in2) [process, below of=in1] {
				$n = 0$
			};
			\node (dec) [decision, below of=in2, yshift=-0.5cm] {
				$n = N_{t}$
			};
			\node (stop) [stop, right of=dec, xshift=2cm] {};
			\node (pr1) [process, below of=dec, yshift=-1cm] {
				\makecell{
					цикл по $j=2..N_{y}-1$ \\
					1) $\alpha_{1} = 1$, $\beta_{1} = -((n+\frac{1}{2})\Delta te^{(j-1)h_{y}})h_{x}$; \\
					2) цикл по $i=2..N_{x}-1$: расчёт $a_{i}$, $b_{i}$, $c_{i}$, $\xi_{i, j}^{n}$; $\alpha_{i}$, $\beta_{i}$; \\
					3) $u_{N_{x}, j}^{n+\frac{1}{2}} = (n+\frac{1}{2})\Delta te^{(j-1)h_{y}}$; \\
					4) цикл по $i=N_{x}-1..1$: $u_{i, j}^{n+\frac{1}{2}} = \alpha_{i}u_{i+1, j}^{n+\frac{1}{2}} + \beta_{i}$.
				}
			};
			\node (pr2) [process, below of=pr1, yshift=-2cm] {
				\makecell{
					цикл по $i=2..N_{x}-1$ \\
					1) $\tilde{\alpha}_{1} = 0$, $\tilde{\beta}_{1} = (n+1)\Delta t(i-1)h_{x}$; \\
					2) цикл по $j=2..N_{y}-1$: расчёт $\tilde{a}_{j}$, $\tilde{b}_{j}$, $\tilde{c}_{j}$, $\tilde{\xi}_{i, j}^{n+\frac{1}{2}}$; $\tilde{\alpha}_{j}$, $\tilde{\beta}_{j}$; \\
					3) $u_{i, N_{y}}^{n+1} = u_{i, N_{y}-1}^{n+1} + (2.7(n+1)\Delta t)h_{y}$; \\
					4) цикл по $j=N_{y}-1..1$: $u_{i, j}^{n+1} = \tilde{\alpha}_{j}u_{i, j+1}^{n+1} + \tilde{\beta}_{j}$.
				}
			};
			\node (pr3) [process, below of=pr2, yshift=-1cm] {
				\makecell{
					цикл по $j=1..N_{y}$ \\
					$u_{1, j}^{n+1} = u_{2, j}^{n+1} + ((n+1)\Delta te^{(j-1)h_{y}})h_{x}$ \\
					$u_{N_{x}, j}^{n+1} = (n+1)\Delta te^{(j-1)h_{y}}$
				}
			};
			\node (pr4) [process, below of=pr3, yshift=-1cm] {
				\makecell{
					цикл по $i=1..N_{x}$ \\
					$u_{i, 1}^{n+1} = (n+1)\Delta t(i-1)h_{x}$ \\
					$u_{i, N_{y}}^{n+1} = u_{i, N_{y}}^{n+1} + ((n+1)\Delta t(i-1)h_{x})h_{y}$
				}
			};
			\node (pr5) [process, below of=pr4, yshift=-1cm] {$n = n + 1$};
			\draw [arrow] (start) -- (in1);
			\draw [arrow] (in1) -- (in2);
			\draw [arrow] (in2) -- (dec);
			\draw [arrow] (dec) -- node[anchor=south] {да} (stop);
			\draw [arrow] (dec) -- node[anchor=west] {нет} (pr1);
			\draw [arrow] (pr1) -- (pr2);
			\draw [arrow] (pr2) -- (pr3);
			\draw [arrow] (pr3) -- (pr4);
			\draw [arrow] (pr4) -- (pr5);
			\draw [arrow] (pr5.west) -- ++(left:8.2cm) -| +(up:16cm) -- (dec.west);
		\end{tikzpicture}
	\end{center}
	
	\subsection*{Задание 15}
	\addcontentsline{toc}{subsection}{Задание 15}
	\large
	Записать схему предиктор-корректор: \par
	Данная схема требует особого способа расщепления интервала $\Delta t$: интервал $\Delta t$ между точками $t^{n}$ и $t^{n+1}$ на разностной сетке делится пополам; интервал $\Delta t/2$ между точками $t^{n}$ и $t^{n+\frac{1}{2}}$ снова делится пополам. \par
	На первом полушаге интервала $\Delta t/2$ записывается неявная разностная схема, в которой учитывается только производная второго порядка по координате \textit{x}:
	\begin{equation}\label{eq:firstPredictor}
		\frac{u_{i, j}^{n+\frac{1}{4}} - u_{i, j}^{n}}{\Delta t/2} = 16\frac{u_{i+1, j}^{n+\frac{1}{4}} - 2u_{i, j}^{n+\frac{1}{4}} + u_{i-1, j}^{n+\frac{1}{4}}}{h_{x}^{2}}.
	\end{equation}
	\par
	На втором полушаге интервала $\Delta t/2$ записывается неявная разностная схема, в которой учитывается только производная второго порядка по координате \textit{y}:
	\begin{equation}\label{eq:secondPredictor}
		\frac{u_{i, j}^{n+\frac{1}{2}} - u_{i, j}^{n+\frac{1}{4}}}{\Delta t/2} = 12\frac{u_{i, j+1}^{n+\frac{1}{2}} - 2u_{i, j}^{n+\frac{1}{2}} + u_{i, j-1}^{n+\frac{1}{2}}}{h_{y}^{2}}.
	\end{equation}
	\par
	Результатом последовательного решения подсхем \eqref{eq:firstPredictor}, \eqref{eq:secondPredictor}, называемых в совокупности \textbf{предиктором}, являются значения функции \textit{u(t, x, y)} на шаге по времени $(n+\frac{1}{2})$. Для завершения расчётов на всём интервале $\Delta t$ используется поправочное разностное соотношение, называемое \textbf{корректором}:
	\small
	\begin{equation}\label{corrector}
		\frac{u_{i, j}^{n+1} - u_{i, j}^{n}}{\Delta t} = 16\frac{u_{i+1, j}^{n+\frac{1}{2}} - 2u_{i, j}^{n+\frac{1}{2}} + u_{i-1, j}^{n+\frac{1}{2}}}{h_{x}^{2}} + 12\frac{u_{i, j+1}^{n+\frac{1}{2}} - 2u_{i, j}^{n+\frac{1}{2}} + u_{i, j-1}^{n+\frac{1}{2}}}{h_{y}^{2}} + (n+\frac{1}{2})\Delta t(i-1)h_{x}e^{(j-1)h_{y}}.
	\end{equation}
	\par
	Таким образом, схема предиктор-корректор в случае двумерных задач состоит из трёх подсхем.

	\subsection*{Задание 16}
	\addcontentsline{toc}{subsection}{Задание 16}
	\large
	Определить порядок аппроксимации разностной схемы:

	\subsection*{Задание 17}
	\addcontentsline{toc}{subsection}{Задание 17}
	\large
	Привести схемы к виду, удобному для использования метода прогонки:
	\subsubsection*{Первая подсхема}
	\large
	Приведу первую подсхему предиктора \eqref{eq:firstPredictor} к виду, удобному для использования метода прогонки:
	\small
	\begin{equation*}
		-\frac{16}{2}\frac{\Delta t}{h_{x}^{2}}u_{i+1, j}^{n+\frac{1}{4}} + (1 + 16\frac{\Delta t}{h_{x}^{2}})u_{i, j}^{n+\frac{1}{4}} - \frac{16}{2}\frac{\Delta t}{h_{x}^{2}}u_{i-1, j}^{n+\frac{1}{4}} = u_{i, j}^{n}.
	\end{equation*}
	\subsubsection*{Вторая подсхема}
	\large
	Приведу вторую подсхему предиктора \eqref{eq:secondPredictor} к виду, удобному для использования метода прогонки:
	\small
	\begin{equation*}
		-\frac{12}{2}\frac{\Delta t}{h_{y}^{2}}u_{i, j+1}^{n+\frac{1}{2}} + (1 + 12\frac{\Delta t}{h_{y}^{2}})u_{i, j}^{n+\frac{1}{2}} - \frac{12}{2}\frac{\Delta t}{h_{y}^{2}}u_{i, j-1}^{n+\frac{1}{2}} = u_{i, j}^{n+\frac{1}{4}}.
	\end{equation*}

	\subsection*{Задание 18}
	\addcontentsline{toc}{subsection}{Задание 18}
	\large
	Проверить сходимость прогонки:
	\subsubsection*{Первая подсхема}
	\large
	Коэффициенты, соответствующие уравнению первой подсхемы предиктора \eqref{eq:firstPredictor}, имеют вид:
	\small
	\begin{center}
		$a_{i}=c_{i}=-\frac{16}{2}\frac{\Delta t}{h_{x}^{2}}$, $\>$ $b_{i}=1 + 16\frac{\Delta t}{h_{x}^{2}}$, $\>$ $\xi_{i, j}^{n}=u_{i, j}^{n}$.
	\end{center}
	\par
	\large
	Легко видеть, что для первой подсхемы предиктора \eqref{eq:secondPredictor} схемы расщепления достаточное условие сходимости прогонки выполняется:
	\begin{equation*}
		\lvert a_{i} \rvert + \lvert c_{i} \rvert = 16\frac{\Delta t}{h_{x}^{2}} < 1 + 16\frac{\Delta t}{h_{x}^{2}} = \lvert b_{i} \rvert.
	\end{equation*}
	\subsubsection*{Вторая подсхема}
	\large
	Коэффициенты, соответствующие уравнению второй подсхемы предиктора \eqref{eq:secondPredictor}, имеют вид:
	\small
	\begin{center}
		$\tilde{a}_{j}=\tilde{c}_{j}=-\frac{12}{2}\frac{\Delta t}{h_{y}^{2}}$, $\>$ $\tilde{b}_{j}=1 + 12\frac{\Delta t}{h_{y}^{2}}$, $\>$ $\tilde{\xi}_{i, j}^{n}=u_{i, j}^{n+\frac{1}{4}}$.
	\end{center}
	\par
	\large
	Легко видеть, что для второй подсхемы предиктора \eqref{eq:secondPredictor} схемы расщепления достаточное условие сходимости прогонки выполняется:
	\begin{equation*}
		\lvert \tilde{a}_{j} \rvert + \lvert \tilde{c}_{j} \rvert = 12\frac{\Delta t}{h_{y}^{2}} < 1 + 12\frac{\Delta t}{h_{y}^{2}} = \lvert \tilde{b}_{j} \rvert.
	\end{equation*}

	\subsection*{Задание 19}
	\addcontentsline{toc}{subsection}{Задание 19}
	\large
	Записать рекуррентное прогоночное соотношение для предиктора:
	\subsubsection*{Первая подсхема}
	\large
	Рекуррентное прогоночное соотношение для первой подсхемы предиктора \eqref{eq:firstPredictor} имеет вид:
	\begin{equation*}
		u_{i, j}^{n+\frac{1}{4}} = \alpha_{i}u_{i+1, j}^{n+\frac{1}{4}} + \beta_{i}.
	\end{equation*}
	\par
	Прогоночные коэффициенты:
	\begin{equation*}
		\alpha_{i} = -\frac{a_{i}}{b_{i} + c_{i}\alpha_{i-1}}, \> \beta_{i} = \frac{\xi_{i, j}^{n} - c_{i}\beta_{i-1}}{b_{i} + c_{i}\alpha_{i-1}}.
	\end{equation*}
	\subsubsection*{Вторая подсхема}
	\large
	Рекуррентное прогоночное соотношение для второй подсхемы предиктора \eqref{eq:secondPredictor} имеет вид:
	\begin{equation*}
		u_{i, j}^{n+\frac{1}{2}} = \tilde{\alpha}_{j}u_{i, j+1}^{n+\frac{1}{2}} + \tilde{\beta}_{i}.
	\end{equation*}
	\par
	Прогоночные коэффициенты:
	\begin{equation*}
		\tilde{\alpha}_{j} = -\frac{\tilde{a}_{j}}{\tilde{b}_{j} + \tilde{c}_{j}\tilde{\alpha}_{j-1}}, \> \tilde{\beta}_{j} = \frac{\tilde{\xi}_{i, j}^{n+\frac{1}{4}} - \tilde{c}_{j}\tilde{\beta}_{j-1}}{\tilde{b}_{j} + \tilde{c}_{j}\tilde{\alpha}_{j-1}}.
	\end{equation*}

	\subsection*{Задание 20}
	\addcontentsline{toc}{subsection}{Задание 20}
	\large
	Записать рекуррентное прогоночное соотношение для корректора \eqref{eq:corrector}:
	\small
	\begin{equation*}
		u_{i, j}^{n+1} = u_{i, j}^{n} + 16\Delta t\frac{u_{i+1, j}^{n+\frac{1}{2}} - 2u_{i, j}^{n+\frac{1}{2}} + u_{i-1, j}^{n+\frac{1}{2}}}{h_{x}^{2}} + 12\Delta t\frac{u_{i, j+1}^{n+\frac{1}{2}} - 2u_{i, j}^{n+\frac{1}{2}} + u_{i, j-1}^{n+\frac{1}{2}}}{h_{y}^{2}} + \Delta t((n+\frac{1}{2})\Delta t(i-1)h_{x}e^{(j-1)h_{y}}).
	\end{equation*}
	\subsection*{Задание 21}
	\addcontentsline{toc}{subsection}{Задание 21}
	\large
	Составить алгоритм (блок-схему) расчёта:
	\small
	\tikzstyle{start} = [circle, draw=black!60, fill=white!5, very thick, minimum size=10mm]
	\tikzstyle{stop} = [circle, color=black!60, fill=black!60, very thick, minimum size=10mm]
	\tikzstyle{process} = [rectangle, minimum width=3cm, minimum height=1cm, text centered, draw=black]
	\tikzstyle{decision} = [diamond, minimum width=3cm, minimum height=1cm, text centered, draw=black]
	\tikzstyle{arrow} = [thick,->,>=stealth]
	\begin{center}
		\begin{tikzpicture}[node distance=2cm]
			\node (start) [start] {};
			\node (in1) [process, below of=start] {
				\makecell{
					Задание начальных условий: \\ цикл по $i=1..N_{x}$; цикл по $j=1..N_{x}$ \\ 
					$u_{i, j}^{0} = 0$
				}
			};
			\node (in2) [process, below of=in1] {
				$n = 0$
			};
			\node (dec) [decision, below of=in2, yshift=-0.5cm] {
				$n = N_{t}$
			};
			\node (stop) [stop, right of=dec, xshift=2cm] {};
			\node (pr1) [process, below of=dec, yshift=-1cm] {
				\makecell{
					цикл по $j=2..N_{y}-1$ \\
					1) $\alpha_{1} = 1$, $\beta_{1} = -((n+\frac{1}{4})\Delta te^{(j-1)h_{y}})h_{x}$; \\
					2) цикл по $i=2..N_{x}-1$: расчёт $a_{i}$, $b_{i}$, $c_{i}$, $\xi_{i, j}^{n}$; $\alpha_{i}$, $\beta_{i}$; \\
					3) $u_{N_{x}, j}^{n+\frac{1}{4}} = (n+\frac{1}{4})\Delta te^{(j-1)h_{y}}$; \\
					4) цикл по $i=N_{x}-1..1$: $u_{i, j}^{n+\frac{1}{4}} = \alpha_{i}u_{i+1, j}^{n+\frac{1}{4}} + \beta_{i}$.
				}
			};
			\node (pr2) [process, below of=pr1, yshift=-2cm] {
				\makecell{
					цикл по $i=2..N_{x}-1$ \\
					1) $\tilde{\alpha}_{1} = 0$, $\tilde{\beta}_{1} = (n+\frac{1}{2})\Delta t(i-1)h_{x}$; \\
					2) цикл по $j=2..N_{y}-1$: расчёт $\tilde{a}_{j}$, $\tilde{b}_{j}$, $\tilde{c}_{j}$, $\tilde{\xi}_{i, j}^{n+\frac{1}{4}}$; $\tilde{\alpha}_{j}$, $\tilde{\beta}_{j}$; \\
					3) $u_{i, N_{y}}^{n+\frac{1}{2}} = u_{i, N_{y}-1}^{n+\frac{1}{2}} + (2.7(n+\frac{1}{2})\Delta t)h_{y}$; \\
					4) цикл по $j=N_{y}-1..1$: $u_{i, j}^{n+\frac{1}{2}} = \tilde{\alpha}_{j}u_{i, j+1}^{n+\frac{1}{2}} + \tilde{\beta}_{j}$.
				}
			};
			\node (pr3) [process, below of=pr2, yshift=-1cm] {
				\makecell{
					цикл по $i=2..N_{x}-1$; цикл по $j=2..N_{y}-1$: \\
					$u_{i, j}^{n+1} = u_{i, j}^{n} + 16\Delta t\frac{u_{i+1, j}^{n+\frac{1}{2}} - 2u_{i, j}^{n+\frac{1}{2}} + u_{i-1, j}^{n+\frac{1}{2}}}{h_{x}^{2}} + 12\Delta t\frac{u_{i, j+1}^{n+\frac{1}{2}} - 2u_{i, j}^{n+\frac{1}{2}} + u_{i, j-1}^{n+\frac{1}{2}}}{h_{y}^{2}} + \Delta t((n+\frac{1}{2})\Delta t(i-1)h_{x}e^{(j-1)h_{y}})$ 
				}
			};
			\node (pr4) [process, below of=pr3, yshift=-0.5cm] {
				\makecell{
					цикл по $j=1..N_{y}$ \\
					$u_{1, j}^{n+1} = u_{2, j}^{n+1} + ((n+1)\Delta te^{(j-1)h_{y}})h_{x}$ \\
					$u_{N_{x}, j}^{n+1} = (n+1)\Delta te^{(j-1)h_{y}}$
				}
			};
			\node (pr5) [process, below of=pr4, yshift=-0.5cm] {
				\makecell{
					цикл по $i=1..N_{x}$ \\
					$u_{i, 1}^{n+1} = (n+1)\Delta t(i-1)h_{x}$ \\
					$u_{i, N_{y}}^{n+1} = u_{i, N_{y}}^{n+1} + ((n+1)\Delta t(i-1)h_{x})h_{y}$
				}
			};
			\node (pr6) [process, below of=pr5, yshift=-0.5cm] {$n = n + 1$};
			\draw [arrow] (start) -- (in1);
			\draw [arrow] (in1) -- (in2);
			\draw [arrow] (in2) -- (dec);
			\draw [arrow] (dec) -- node[anchor=south] {да} (stop);
			\draw [arrow] (dec) -- node[anchor=west] {нет} (pr1);
			\draw [arrow] (pr1) -- (pr2);
			\draw [arrow] (pr2) -- (pr3);
			\draw [arrow] (pr3) -- (pr4);
			\draw [arrow] (pr4) -- (pr5);
			\draw [arrow] (pr5) -- (pr6);
			\draw [arrow] (pr6.west) -- ++(left:8.2cm) -| +(up:17.5cm) -- (dec.west);
		\end{tikzpicture}
	\end{center}
\end{document}
