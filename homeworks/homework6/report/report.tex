\documentclass[12pt, a4paper]{report}
\usepackage[top=1cm, left=1cm, right=1cm]{geometry}

\usepackage[utf8]{inputenc}
\usepackage[russian]{babel}

\usepackage{array}
\newcolumntype{M}[1]{>{\centering\arraybackslash}m{#1}}

\usepackage{hyperref}
\hypersetup{
	colorlinks,
	citecolor=black,
	filecolor=black,
	linkcolor=black,
	urlcolor=black
}

\usepackage{sectsty}
\allsectionsfont{\centering}

\usepackage{indentfirst}
\setlength\parindent{24pt}

\usepackage{makecell}

\usepackage{amsmath}

\usepackage{tikz}
\usetikzlibrary{shapes.geometric, arrows}
\usetikzlibrary{arrows.meta}
\usetikzlibrary{decorations.text}

\begin{document}
	\begin{titlepage}
		\begin{center}
			\large \textbf{Министерство науки и высшего образования Российской Федерации} \\
			\large \textbf{Федеральное государственное бюджетное образовательное учреждение высшего образования} \\
			\large \textbf{«Российский химико-технологический университет имени Д.И. Менделеева»} \\

			\vspace*{4cm}
			\LARGE \textbf{ОТЧЕТ ПО ДОМАШНЕЙ РАБОТЕ №6}

			\vspace*{4cm}
			\begin{flushright}
				\Large
				\begin{tabular}{>{\raggedleft\arraybackslash}p{9cm} p{10cm}}
					Выполнил студент группы КС-36: & Золотухин А.А. \\
					Ссылка на репозиторий: & https://github.com/ \\
					& CorgiPuppy/ \\
					& num-methods-eq-math-phys-chem-labs \\
					Приняла: & Кольцова Элеонора Моисеевна \\
					Дата сдачи: & 31.03.2025 \\
				\end{tabular}
			\end{flushright}

			\vspace*{6cm}
			\Large \textbf{Москва \\ 2025}
		\end{center}
	\end{titlepage}

	\tableofcontents
	\thispagestyle{empty}
	\newpage

	\pagenumbering{arabic}

	\section*{Описание задачи}
	\addcontentsline{toc}{section}{Описание задачи}
	\large
	\begin{center}
		\begin{tabular}{||c|c|c||}
			\hline
			Уравнение & Интервалы переменных & Начальные и граничные условия \\
			\hline
			$ \frac{\partial u}{\partial t} - 8\frac{\partial u}{\partial x} = x^{2} - 1 $ & \makecell{$ x \in [0, 1] $ \\ $ t \in [0, 1] $} & \makecell{$ u(t = 0, x) = x $ \\ $ u(t, x = 0) = t $ \\ $ u(t, x = 1) = t $} \\

			\hline
		\end{tabular}
	\end{center}
	\par
	Для заданного уравнения:
	\begin{enumerate}
		\item записать явную разностную схему;
		\item проверить условие устойчивости разностной схемы;
		\item вывести рекуррентное соотношение;
		\item составить алгоритм (блок-схему) расчёта;
		\item записать неявную разностную схему;
		\item проверить условие устойчивости разностной схемы; 
		\item вывести рекуррентное соотношение;
		\item составить алгоритм (блок-схему) расчёта;
	\end{enumerate}

	\newpage

	\section*{Выполнение задачи}
	\addcontentsline{toc}{section}{Выполнение задачи}

	\subsection*{Задание 1}
	\addcontentsline{toc}{subsection}{Задание 1}
	\large
	Записать явную разностную схему:
	\begin{equation}\label{eq:explicit}
		\frac{u_{j}^{n+1}-u_{j}^{n}}{\Delta t} - 8\frac{u_{j+1}^{n}-u_{j}^{n}}{h} = ((j - 1)h)^{2} - 1.
	\end{equation}

	\subsection*{Задание 2}
	\addcontentsline{toc}{subsection}{Задание 2}
	\large
	Проверить условие устойчивости разностной схемы:
	Исследую устойчивость разностной схемы \eqref{eq:explicit} с помощью спектрального метода. Для этого отброшу член $((j - 1)h)^{2} - 1$, наличие которого не оказывает влияния на устойчивость разностной схемы, и представлю решение в виде гармоники:
	\begin{equation}\label{eq:harmonic}
		u_{j}^{n} = \lambda^{n}e^{i \alpha j}.
	\end{equation}
	\par
	Подставляя \eqref{eq:harmonic} в \eqref{eq:explicit}:
	\begin{equation*}
		\frac{\lambda^{n+1}e^{i \alpha j}-\lambda^{n}e^{i \alpha j}}{\Delta t} - 8\frac{\lambda^{n}e^{i \alpha (j + 1)}-\lambda^{n}e^{i \alpha j}}{h} = 0.
	\end{equation*}
	\par
	Упрощаю данное выражение, деля левую и правую его части на $\lambda^{n}e^{i \alpha j}$, и выражаю $\lambda$:
	\begin{equation*}
		\frac{\lambda - 1}{\Delta t} - 8\frac{e^{i \alpha} - 1}{h} = 0 \Rightarrow \lambda = 1 - 8\frac{\Delta t}{h} + 8\frac{\Delta t}{h}e^{i \alpha}.
	\end{equation*}
	\par
	Комплексный вид полученного выражения свидетельствует о том, что необходимое условие устойчивости разностных схем также следует рассматривать в применении к комплексным числам. То есть, неравенство
	\begin{equation}\label{eq:stability}
		\lvert \lambda \rvert \leq 1 
	\end{equation}
	означает, что для того, чтобы разностная схема была устойчива, необходимо чтобы собственные числа оператора перехода были расположены внутри или на границе круга радиусом 1, центр которого находится в начале координат комплексной плоскости (рис. \ref{fig:stability}).
	\begin{figure}
		\centering
		\begin{tikzpicture}[scale=1.6]
			\draw (-3, 0) -- (3, 0) node[right] {Re};
			\draw (0, -3) -- (0, 3) node[above] {Im};
			\foreach \x/\xtext in {-2, -1, 1, 2}
			\draw (\x cm,1pt) -- (\x cm,-1pt) node[anchor=north,fill=none] {$\xtext$};
			\foreach \y/\ytext in {-2/-2i, -1/-i, 1/i, 2/2i}
			\draw (1pt,\y cm) -- (-1pt,\y cm) node[anchor=east,fill=none] {$\ytext$};
			
			\draw (0,0) circle [radius=1cm];
			\draw[postaction={decorate,
				decoration={raise=4pt, text along path,
				text align=center,
				text={1}}}, thick, -Stealth] (0,0) -- (0.7,0.7);
		\end{tikzpicture}
		\caption{Графическая интерпретация условия устойчивости \eqref{eq:stability}}
		\label{fig:stability}
	\end{figure}
	\par
	Введу следующие обозначение:
	\begin{equation*}
		r = 8\frac{\Delta t}{h} > 0 \Rightarrow \lambda = 1 - r + r e^{i \alpha}.
	\end{equation*}
	\par
	Полученное выражение свидетельствует о том, что собственные числа оператора расположены на комплексной плоскости на окружности с центром в точке $(1 - r,0)$ и радиусом:
	\begin{equation*}
		\lvert re^{i \alpha} \rvert = \lvert r\cos{\alpha}+ir\sin{\alpha} \rvert = \sqrt{r^{2}\cos^{2}{\alpha}+r^{2}\sin^{2}{\alpha}} = r.
	\end{equation*}
	\par
	Сравнивая расположение этой окружности на комплексной плоскости с условием \eqref{eq:stability}, получаю три различных варианта (рис. \ref{fig:rless1}-\ref{fig:rgreater1}). Видно, что окружность, соответствующая собственным числам оператора перехода, при $r < 1$ находится внутри круга, соответствующего условию \eqref{eq:stability}; при $r > 1$ - вне этого круга, а при $r = 1$ совпадает с его границей. Таким образом, разностная схема \eqref{eq:explicit} будет \textbf{условно устойчива} при выполнении следующего условия:
	\begin{equation*}
		r = 8\frac{\Delta t}{h} \leq 1
	\end{equation*}
	\begin{figure}
		\centering
		\begin{tikzpicture}[scale=1.6]
			\draw (-3, 0) -- (3, 0) node[right] {Re};
			\draw (0, -3) -- (0, 3) node[above] {Im};
			\foreach \x/\xtext in {-2, -1, 1, 2}
			\draw (\x cm,1pt) -- (\x cm,-1pt) node[anchor=north,fill=none] {$\xtext$};
			\foreach \y/\ytext in {-2/-2i, -1/-i, 1/i, 2/2i}
			\draw (1pt,\y cm) -- (-1pt,\y cm) node[anchor=east,fill=none] {$\ytext$};
			
			\draw (0,0) circle [radius=1cm];
			\draw[postaction={decorate,
				decoration={raise=4pt, text along path,
				text align=center,
				text={1}}}, thick, -Stealth] (0,0) -- (0.7,0.7);
			\draw [blue, very thick] (1-0.4,0) circle [radius=0.4cm];
			\filldraw [black] (1-0.4,0) circle [radius=0.5mm];
			\draw[postaction={decorate,
				decoration={raise=4pt, text along path,
				text align=center, reverse path,
				text={r}}}, thick, -Stealth] (1-0.4,0) -- (1-0.5,0.4);
		\end{tikzpicture}
		\caption{Исследование устойчивости разностной схемы \eqref{eq:explicit} при $r < 1$}
		\label{fig:rless1}
	\end{figure}
	\begin{figure}
		\centering
		\begin{tikzpicture}[scale=1.6]
			\draw (-3, 0) -- (3, 0) node[right] {Re};
			\draw (0, -3) -- (0, 3) node[above] {Im};
			\foreach \x/\xtext in {-2, -1, 1, 2}
			\draw (\x cm,1pt) -- (\x cm,-1pt) node[anchor=north,fill=none] {$\xtext$};
			\foreach \y/\ytext in {-2/-2i, -1/-i, 1/i, 2/2i}
			\draw (1pt,\y cm) -- (-1pt,\y cm) node[anchor=east,fill=none] {$\ytext$};
			
			\draw (0,0) circle [radius=1cm];
			\draw[postaction={decorate,
				decoration={raise=4pt, text along path,
				text align=center,
				text={1}}}, thick, -Stealth] (0,0) -- (0.7,0.7);
			\draw [blue, very thick] (0,0) circle [radius=1cm];
			\filldraw [black] (0,0) circle [radius=0.5mm];
			\draw[postaction={decorate,
				decoration={raise=4pt, text along path,
				text align=center, reverse path,
				text={r}}}, thick, -Stealth] (0,0) -- (-0.7,0.7);
		\end{tikzpicture}
		\caption{Исследование устойчивости разностной схемы \eqref{eq:explicit} при $r = 1$}
		\label{fig:requal1}
	\end{figure}
	\begin{figure}
		\centering
		\begin{tikzpicture}[scale=1.6]
			\draw (-3, 0) -- (3, 0) node[right] {Re};
			\draw (0, -3) -- (0, 3) node[above] {Im};
			\foreach \x/\xtext in {-2, -1, 1, 2}
			\draw (\x cm,1pt) -- (\x cm,-1pt) node[anchor=north,fill=none] {$\xtext$};
			\foreach \y/\ytext in {-2/-2i, -1/-i, 1/i, 2/2i}
			\draw (1pt,\y cm) -- (-1pt,\y cm) node[anchor=east,fill=none] {$\ytext$};
			
			\draw (0,0) circle [radius=1cm];
			\draw[postaction={decorate,
				decoration={raise=4pt, text along path,
				text align=center,
				text={1}}}, thick, -Stealth] (0,0) -- (0.7,0.7);
			\draw [blue, very thick] (1-1.4,0) circle [radius=1.4cm];
			\filldraw [black] (1-1.4,0) circle [radius=0.5mm];
			\draw[postaction={decorate,
				decoration={raise=4pt, text along path,
				text align=center, reverse path,
				text={r}}}, thick, -Stealth] (1-1.4,0) -- (1-1.7,1.4);
		\end{tikzpicture}
		\caption{Исследование устойчивости разностной схемы \eqref{eq:explicit} при $r > 1$}
		\label{fig:rgreater1}
	\end{figure}

	\subsection*{Задание 3}
	\addcontentsline{toc}{subsection}{Задание 3}
	\large
	Вывести рекуррентное соотношение:
	Выражаю $u_{j}^{n+1}$ из разностной схемы \eqref{eq:explicit}:
	\begin{equation*}
		u_{j}^{n+1} = u_{j}^{n}
	\end{equation*}
\end{document}
