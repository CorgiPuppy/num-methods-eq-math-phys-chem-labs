\documentclass[12pt, a4paper]{report}
\usepackage[top=1cm, left=1cm, right=1cm]{geometry}

\usepackage[utf8]{inputenc}
\usepackage[russian]{babel}

\usepackage{array}
\newcolumntype{M}[1]{>{\centering\arraybackslash}m{#1}}

\usepackage{hyperref}
\hypersetup{
	colorlinks,
	citecolor=black,
	filecolor=black,
	linkcolor=black,
	urlcolor=black
}

\usepackage{sectsty}
\allsectionsfont{\centering}

\usepackage{indentfirst}
\setlength\parindent{24pt}

\usepackage{makecell}

\usepackage{amsmath}

\def\H{\rule{0pt}{1.5ex}H}

\usepackage{tikz}
\usetikzlibrary{shapes.geometric, arrows}
\usetikzlibrary{arrows.meta}
\usetikzlibrary{decorations.text}

\begin{document}
	\begin{titlepage}
		\begin{center}
			\large \textbf{Министерство науки и высшего образования Российской Федерации} \\
			\large \textbf{Федеральное государственное бюджетное образовательное учреждение высшего образования} \\
			\large \textbf{«Российский химико-технологический университет имени Д.И. Менделеева»} \\

			\vspace*{4cm}
			\LARGE \textbf{ОТЧЕТ ПО ДОМАШНЕЙ РАБОТЕ №12}

			\vspace*{4cm}
			\begin{flushright}
				\Large
				\begin{tabular}{>{\raggedleft\arraybackslash}p{9cm} p{10cm}}
					Выполнил студент группы КС-36: & Золотухин Андрей Александрович \\
					Ссылка на репозиторий: & https://github.com/ \\
					& CorgiPuppy/ \\
					& num-methods-eq-math-phys-chem-labs \\
					Приняла: & Кольцова Элеонора Моисеевна \\
					Дата сдачи: & 12.05.2025 \\
				\end{tabular}
			\end{flushright}

			\vspace*{6cm}
			\Large \textbf{Москва \\ 2025}
		\end{center}
	\end{titlepage}

	\tableofcontents
	\thispagestyle{empty}
	\newpage

	\pagenumbering{arabic}

	\section*{Описание задачи}
	\addcontentsline{toc}{section}{Описание задачи}
	\large
	\begin{center}
		\begin{tabular}{||c|c|c|c||}
			\hline
			Уравнение & Интервал переменной & Метод & Граничные условия \\

			\hline
			$ \frac{du}{dx}=\frac{d^{2}u}{dx^{2}}+x$ & $x \in [0, 1]$ & \makecell{Установление с \\ явной схемой; \\ Установление с \\ неявной схемой; \\ Установление со \\ схемой Кранка- \\ Николсона} & \makecell{$ \begin{cases} u(x = 0) = 0 \\ \frac{du}{dx}(x = 1) = 1 \end{cases}$} \\

			\hline
		\end{tabular}
	\end{center}
	\par
	Для заданного уравнения:
	\begin{enumerate}
		\item представить задачу в нестационарном виде;
		\item записать явную разностную схему;
		\item вывести рекуррентное соотношение;
		\item составить алгоритм (блок-схему) расчёта;
		\item записать неявную разностную схему;
		\item привести схему к виду, удобному для использования метода прогонки;
		\item проверить сходимость прогонки;
		\item найти $\alpha_1$, $\beta_1$, $u_{N}^{n+1}$;
		\item записать рекуррентное прогоночное соотношение;
		\item составить алгоритм (блок-схему) расчёта;
		\item записать разностную схему Кранка-Николсона;
		\item привести схему к виду, удобному для использования метода прогонки;
		\item проверить сходимость прогонки;
		\item найти $\alpha_1$, $\beta_1$, $u_{N}^{n+1}$;
		\item записать рекуррентное прогоночное соотношение;
		\item составить алгоритм (блок-схему) расчёта;
	\end{enumerate}

	\begin{center}
		\begin{tabular}{||c|c|c|c||}
			\hline
			Уравнение & Интервал переменной & Метод & Граничные условия \\

			\hline
			$ -\frac{du}{dx}=4\frac{d^{2}u}{dx^{2}}+2x$ & $x \in [0, 1]$ & \makecell{Установление с \\ явной схемой; \\ Установление со \\ схемой Кранка- \\ Николсона} & \makecell{$ \begin{cases} \frac{du}{dx}(x = 0) = 2 \\ u(x = 1) = 2 \end{cases}$} \\

			\hline
		\end{tabular}
	\end{center}
	\par
	Для заданного уравнения:
	\begin{enumerate}
		\setcounter{enumi}{16}
		\item представить задачу в нестационарном виде;
		\item записать явную разностную схему;
		\item вывести рекуррентное соотношение;
		\item составить алгоритм (блок-схему) расчёта;
		\item записать разностную схему Кранка-Николсона;
		\item привести схему к виду, удобному для использования метода прогонки;
		\item проверить сходимость прогонки;
		\item найти $\alpha_1$, $\beta_1$, $u_{N}^{n+1}$;
		\item записать рекуррентное прогоночное соотношение;
		\item составить алгоритм (блок-схему) расчёта;
	\end{enumerate}

	\newpage

	\section*{Выполнение задачи}
	\addcontentsline{toc}{section}{Выполнение задачи}

	\subsection*{Задание 1}
	\addcontentsline{toc}{subsection}{Задание 1}
	\large
	Представить задачу в нестационарном виде: \par
	Представлю стационарную задачу в нестационарном виду. Для этого в уравнение необходимо добавить фиктивную производную по времени:
	\begin{equation}\label{eq:nonstationary}
		\frac{du}{dx}=\frac{d^{2}u}{dx^{2}}+x \rightarrow \frac{\partial \tilde{u}}{\partial \tau}+\frac{\partial \tilde{u}}{\partial x}=\frac{\partial^{2} \tilde{u}}{\partial x^{2}}+x.
	\end{equation}
	\par
	При этом искомая функция станет уже функцией двух переменных:
	\begin{equation*}
		u(x) \rightarrow \tilde{u}(x,\tau).
	\end{equation*}
	
	\subsection*{Задание 2}
	\addcontentsline{toc}{subsection}{Задание 2}
	\large
	Записать явную разностную схему для уравнения \eqref{eq:nonstationary}:
	\begin{equation}\label{eq:explicit}
		\frac{u_{j}^{n+1} - u_{j}^{n}}{\Delta t} + \frac{u_{j}^{n} - u_{j-1}^{n}}{h} = \frac{u_{j+1}^{n} - u_{j}^{n} + u_{j-1}^{n}}{h^{2}} + (j-1)h.
	\end{equation}

	\subsection*{Задание 3}
	\addcontentsline{toc}{subsection}{Задание 3}
	\large
	Вывести рекуррентное соотношение для уравнения \eqref{eq:explicit}: \par
	Выражая из разностной схемы \eqref{eq:explicit} величину $u_{j}^{n+1}$, получаю рекуррентное соотношение
	\begin{equation*}
		u_{j}^{n+1} = u_{j}^{n} + \Delta t[\frac{1}{h}(u_{j-1}^{n} - u_{j}^{n}) + \frac{1}{h^{2}}(u_{j+1}^{n} - u_{j}^{n} + u_{j-1}^{n}) + (j-1)h],
	\end{equation*}
	которое с учётом равенства $\Delta t = \frac{h^{2}}{h+2}$ преобразуется к виду:
	\begin{equation*}
		\makecell{
			u_{j}^{n+1} = u_{j}^{n} + \frac{h^{2}}{h+2}[\frac{1}{h^{2}}u_{j+1}^{n} - (2\frac{1}{h^{2}}+\frac{1}{h})u_{j}^{n} + (\frac{1}{h^{2}}+\frac{1}{h})u_{j-1}^{n} + (j-1)h] \Rightarrow \\
			\Rightarrow u_{j}^{n+1} = \frac{u_{j+1}^{n} + (1+h)u_{j-1}^{n} + h^{2}(j-1)h}{h+2}.
		}
	\end{equation*}

	\subsection*{Задание 4}
	\addcontentsline{toc}{subsection}{Задание 4}
	\large
	Составить алгоритм (блок-схему) расчёта:

	\subsection*{Задание 5}
	\addcontentsline{toc}{subsection}{Задание 5}
	\large
	Записать неявную разностную схему для уравнения \eqref{eq:nonstationary}:
	\begin{equation}\label{eq:implicit}
		\frac{u_{j}^{n+1} - u_{j}^{n}}{\Delta t} + \frac{u_{j}^{n+1} - u_{j-1}^{n+1}}{h} = \frac{u_{j+1}^{n+1} - u_{j}^{n+1} + u_{j-1}^{n+1}}{h^{2}} + (j-1)h.
	\end{equation}

	\subsection*{Задание 6}
	\addcontentsline{toc}{subsection}{Задание 6}
	\large
	Привести схему \eqref{eq:implicit} к виду, удобному для использования метода прогонки:
	\small
	\begin{equation*}
		-\frac{\Delta t}{h^{2}}u_{j+1}^{n+1} + (1+\frac{\Delta t}{h} + 2\frac{\Delta t}{h^{2}})u_{j}^{n+1} - (\frac{\Delta t}{h} + \frac{\Delta t}{h^{2}})u_{j-1}^{n+1} = u_{j}^{n} + \Delta t(j-1)h.
	\end{equation*}
	\large
	\par
	Введу следующие обозначения:
	\begin{equation*}
		\makecell{
			a_{j}=-\frac{\Delta t}{h^{2}}, \> b_{j}=1+\frac{\Delta t}{h} + 2\frac{\Delta t}{h^{2}}, \> c_{j}=-(\frac{\Delta t}{h} + \frac{\Delta t}{h^{2}}), \\
			\xi_{j}^{n}=u_{j}^{n} + \Delta t(j-1)h.
		}
	\end{equation*}
	\par
	С учётом обозначений равенство будет иметь вид:
	\begin{equation*}
		\alpha_{j}u_{j+1}^{n+1} + b_{j}u_{j}^{n+1} + c_{j}u_{j-1}^{n+1} = \xi_{j}^{n}.
	\end{equation*}
	\par
	Данное преобразование называется \textit{преобразованием} \textit{неявной} \textit{схемы к виду}, \textit{удобному для} \textit{использования метода прогонки}.

	\subsection*{Задание 7}
	\addcontentsline{toc}{subsection}{Задание 7}
	\large
	Проверить сходимость прогонки: \par
	Легко видеть, что для разностной схемы \eqref{eq:implicit} достаточное условие сходимости прогонки выполняется:
	\begin{equation*}
		\lvert a_{j} \rvert + \lvert c_{j} \rvert = \frac{\Delta t}{h} + 2\frac{\Delta t}{h^{2}} < 1 + \frac{\Delta t}{h} + 2\frac{\Delta t}{h^{2}} = \lvert b_{j} \rvert.
	\end{equation*}

	\subsection*{Задание 8}
	\addcontentsline{toc}{subsection}{Задание 8}
	\large
	Найти $\alpha_1$, $\beta_1$, $u_{N}^{n+1}$: \par
	Для реализации неявной разностной схемы требуется ввести некоторое дополнительное условие, связывающее значения функции \textit{u(t, x)} на \textit{(n+1)}-м шаге по времени. Представлю это дополнительное условие в виде линейной зависимости
	\begin{equation}\label{eq:recurrentImplicit}
		u_{j}^{n+1} = \alpha_{j} u_{j+1}^{n+1} + \beta{j},
	\end{equation}
	справедливой для любого из значений $j=1..N-1$. \par
	Соотношение \eqref{eq:recurrentImplicit} называют \textbf{рекуррентным прогоночным соотношением}, а коэффициенты $\alpha_{j}$, $\beta_{j}$ - \textbf{прогоночными коэффициентами}. \\
	\par
	Для определения прогоночных коэффициентов на \textit{1}-м шаге по координате \textit{x}, использую рекуррентное прогоночное соотношение \eqref{eq:recurrentImplicit}, записанное для $j=1$:
	\begin{equation*}
		u_{1}^{n+1} = \alpha_{1} u_{2}^{n+1} + \beta_{1}
	\end{equation*}
	и левое граничное условие:
	\begin{equation*}
		u_{1}^{n+1} = 0.
	\end{equation*}
	\par
	Сравнивая эти два соотношения, получаю:
	\begin{equation*}
		\alpha_{1} = 0, \beta_{1} = 0.
	\end{equation*}
	\par
	Значение функции \textit{u(t, x)} на \textit{(n+1)}-м шаге по времени в крайней правой точке, которое можно определить из правого граничного условия:
	\begin{equation*}
		u_{N}^{n+1} = u_{N-1}^{n+1} + h.
	\end{equation*}	
	\par
	Используя рекуррентное прогоночное соотношение \eqref{eq:recurrentCrankNicholson}, записанное для $j=N-1$:
	\begin{equation*}
		u_{N-1}^{n+1} = \alpha_{N-1} u_{N}^{n+1} + \beta_{N-1}
	\end{equation*}
	и подставив его в значение функции \textit{u(t, x)} на \textit{(n+1)}-м шаге по времени в крайней правой точке, можно записать в ином виде:
	\begin{equation*}	
		\makecell{
			u_{N}^{n+1} = \alpha_{N-1} u_{N}^{n+1} + \beta_{N-1} + h \Rightarrow \\
			\Rightarrow u_{N}^{n+1} = \frac{\beta_{N-1} + h}{1 - \alpha_{N-1}}.
		}
	\end{equation*}

	\subsection*{Задание 9}
	\addcontentsline{toc}{subsection}{Задание 9}
	\large
	Записать рекуррентное прогоночное соотношение: \par
	Соотношение \eqref{eq:recurrentImplicit} является \textbf{рекуррентным прогоночным соотношением}.
		
	\subsection*{Задание 10}
	\addcontentsline{toc}{subsection}{Задание 10}
	\large
	Составить алгоритм (блок-схему) расчёта:
	\tikzstyle{start} = [circle, draw=black!60, fill=white!5, very thick, minimum size=13mm]
	\tikzstyle{stop} = [circle, draw=black!60, fill=white!60, very thick, minimum size=13mm]
	\tikzstyle{process} = [rectangle, minimum width=3cm, minimum height=1cm, text centered, draw=black]
	\tikzstyle{decision} = [diamond, minimum width=3cm, minimum height=1cm, text centered, draw=black]
	\tikzstyle{arrow} = [thick,->,>=stealth]
	\begin{center}
		\begin{tikzpicture}[node distance=2cm]
			\node (start) [start] {Start};
			\node (in1) [process, below of=start, yshift=-1cm] {
				\makecell{
					Задание начальных условий: \\
					цикл по $j=1..N$:\\ 
					$u_{j}^{0} = (j-1)h$
				}
			};
			\node (in2) [process, below of=in1] {
				$n = 0$
			};
			\node (pr1) [process, below of=in2, yshift=-0.5cm] {
				\makecell{
					Определение $\alpha_{1}$, $\beta_{1}$ из левого граничного условия \\
					$\alpha_{1} = 0$ \\
					$\beta_{1} = 0$
				}
			};
			\node (pr2) [process, below of=pr1, yshift=-1cm] {
				\makecell{
					цикл по $j=2..N-1$: \\
					расчёт $a_{j}$, $b_{j}$, $c_{j}$, $\xi_{j}^{n}$; $\alpha_{j}=-\frac{\alpha{j}}{b_{j}+c_{j}\alpha_{j-1}}$, $\beta_{j}=\frac{\xi_{j}^{n}-c_{j}\beta{j-1}}{b_{j}+c_{j}\alpha_{j-1}}$
				}
			};
			\node (pr3) [process, below of=pr2, yshift=-0.5cm] {
				\makecell{
					Определение $u_{N}^{n+1}$ из правого граничного условия \\
					$u_{N}^{n+1} = u_{N}^{n+1} = \frac{\beta_{N-1} + h}{1 - \alpha_{N-1}}$
				}
			};
			\node (pr4) [process, below of=pr3, yshift=-0.5cm] {
				\makecell{
					цикл по $j=N-1..1$: \\
					$u_{j}^{n+1} = \alpha_{j}u_{j+1}^{n+1} + \beta_{j}$
				}
			};
			\node (dec) [decision, below of=pr4, yshift=-2cm] {
				$\lvert \lvert u^{n+1} - u^{n} \rvert \rvert \le \epsilon$
			};
			\node (stop) [stop, right of=dec, xshift=2cm] {End};
			\node (pr5) [process, below of=dec, yshift=-2cm] {
				$n = n + 1$
			};
			\draw [arrow] (start) -- (in1);
			\draw [arrow] (in1) -- (in2);
			\draw [arrow] (in2) -- (pr1);
			\draw [arrow] (pr1) -- (pr2);
			\draw [arrow] (pr2) -- (pr3);
			\draw [arrow] (pr3) -- (pr4);
			\draw [arrow] (pr4) -- (dec);
			\draw [arrow] (dec) -- node[anchor=west] {нет} (pr5);
			\draw [arrow] (dec) -- node[anchor=south] {да} (stop);
			\draw [arrow] (pr5.west) -- ++(left:8.2cm) -| +(up:16cm) -- (pr1.west);
		\end{tikzpicture}
	\end{center}

	\subsection*{Задание 11}
	\addcontentsline{toc}{subsection}{Задание 11}
	\large
	Записать разностную схему Кранка-Николсона для уравнения \eqref{eq:nonstationary}:
	\normalsize
	\begin{equation}\label{eq:crankNicholson}
		\frac{u_{j}^{n+1} - u_{j}^{n}}{\Delta t} + \frac{1}{2}\frac{u_{j}^{n+1} - u_{j-1}^{n+1}}{h} + \frac{1}{2}\frac{u_{j}^{n} - u_{j-1}^{n}}{h} = \frac{1}{2}\frac{u_{j+1}^{n+1} - u_{j}^{n+1} + u_{j-1}^{n+1}}{h^{2}} + \frac{1}{2}\frac{u_{j+1}^{n} - u_{j}^{n} + u_{j-1}^{n}}{h^{2}} + (j-1)h.
	\end{equation}

	\subsection*{Задание 12}
	\addcontentsline{toc}{subsection}{Задание 12}
	\large
	Привести схему \eqref{eq:crankNicholson} к виду, удобному для использования метода прогонки:
	\small
	\begin{equation*}
		-\frac{1}{2}\frac{\Delta t}{h^{2}}u_{j+1}^{n+1} + (1+\frac{1}{2}\frac{\Delta t}{h} + \frac{\Delta t}{h^{2}})u_{j}^{n+1} - (\frac{1}{2}\frac{\Delta t}{h} + \frac{1}{2}\frac{\Delta t}{h^{2}})u_{j-1}^{n+1} = u_{j}^{n} + \frac{1}{2}\frac{\Delta t}{h^{2}}(u_{j+1}^{n}-2u_{j}^{n}+u_{j-1}^{n}) - \frac{1}{2}\frac{\Delta t}{h}(u_{j}^{n}-u_{j-1}^{n}) + \Delta t (j-1)h.
	\end{equation*}
	\large
	\par
	Введу следующие обозначения:
	\begin{equation*}
		\makecell{
			a_{j}=-\frac{1}{2}\frac{\Delta t}{h^{2}}, \> b_{j}=1+\frac{1}{2}\frac{\Delta t}{h} + \frac{\Delta t}{h^{2}}, \> c_{j}=-(\frac{1}{2}\frac{\Delta t}{h} + \frac{1}{2}\frac{\Delta t}{h^{2}}), \\
			\xi_{j}^{n}=u_{j}^{n} + \frac{1}{2}\frac{\Delta t}{h^{2}}(u_{j+1}^{n}-2u_{j}^{n}+u_{j-1}^{n}) - \frac{1}{2}\frac{\Delta t}{h}(u_{j}^{n}-u_{j-1}^{n}) + \Delta t (j-1)h.
		}
	\end{equation*}
	\par
	С учётом обозначений равенство будет иметь вид:
	\begin{equation*}
		\alpha_{j}u_{j+1}^{n+1} + b_{j}u_{j}^{n+1} + c_{j}u_{j-1}^{n+1} = \xi_{j}^{n}.
	\end{equation*}
	\par
	Данное преобразование называется \textit{преобразованием} \textit{схемы Кранка-Николсона} \textit{к виду}, \textit{удобному для} \textit{использования метода прогонки}.

	\subsection*{Задание 13}
	\addcontentsline{toc}{subsection}{Задание 13}
	\large
	Проверить сходимость прогонки: \par
	Легко видеть, что для разностной схемы \eqref{eq:crankNicholson} достаточное условие сходимости прогонки выполняется:
	\begin{equation*}
		\lvert a_{j} \rvert + \lvert c_{j} \rvert = \frac{1}{2}\frac{\Delta t}{h} + \frac{\Delta t}{h^{2}} < 1 + \frac{1}{2}\frac{\Delta t}{h} + \frac{\Delta t}{h^{2}} = \lvert b_{j} \rvert.
	\end{equation*}

	\subsection*{Задание 14}
	\addcontentsline{toc}{subsection}{Задание 14}
	\large
	Найти $\alpha_1$, $\beta_1$, $u_{N}^{n+1}$: \par
	Для реализации разностной схемы Кранка-Николсона требуется ввести некоторое дополнительное условие, связывающее значения функции \textit{u(t, x)} на \textit{(n+1)}-м шаге по времени. Представлю это дополнительное условие в виде линейной зависимости
	\begin{equation}\label{eq:recurrentCrankNicholson}
		u_{j}^{n+1} = \alpha_{j} u_{j+1}^{n+1} + \beta{j},
	\end{equation}
	справедливой для любого из значений $j=1..N-1$. \par
	Соотношение \eqref{eq:recurrentCrankNicholson} называют \textbf{рекуррентным прогоночным соотношением}, а коэффициенты $\alpha_{j}$, $\beta_{j}$ - \textbf{прогоночными коэффициентами}. \\
	\par
	Для определения прогоночных коэффициентов на \textit{1}-м шаге по координате \textit{x}, использую рекуррентное прогоночное соотношение \eqref{eq:recurrentCrankNicholson}, записанное для $j=1$:
	\begin{equation*}
		u_{1}^{n+1} = \alpha_{1} u_{2}^{n+1} + \beta_{1}
	\end{equation*}
	и левое граничное условие:
	\begin{equation*}
		u_{1}^{n+1} = 0.
	\end{equation*}
	\par
	Сравнивая эти два соотношения, получаю:
	\begin{equation*}
		\alpha_{1} = 0, \beta_{1} = 0.
	\end{equation*}
	\par
	Значение функции \textit{u(t, x)} на \textit{(n+1)}-м шаге по времени в крайней правой точке, которое можно определить из правого граничного условия:
	\begin{equation*}
		u_{N}^{n+1} = u_{N-1}^{n+1} + h.
	\end{equation*}	
	\par
	Используя рекуррентное прогоночное соотношение \eqref{eq:recurrentCrankNicholson}, записанное для $j=N-1$:
	\begin{equation*}
		u_{N-1}^{n+1} = \alpha_{N-1} u_{N}^{n+1} + \beta_{N-1}
	\end{equation*}
	и подставив его в значение функции \textit{u(t, x)} на \textit{(n+1)}-м шаге по времени в крайней правой точке, можно записать в ином виде:
	\begin{equation*}	
		\makecell{
			u_{N}^{n+1} = \alpha_{N-1} u_{N}^{n+1} + \beta_{N-1} + h \Rightarrow \\
			\Rightarrow u_{N}^{n+1} = \frac{\beta_{N-1} + h}{1 - \alpha_{N-1}}.
		}
	\end{equation*}

	\subsection*{Задание 15}
	\addcontentsline{toc}{subsection}{Задание 15}
	\large
	Записать рекуррентное прогоночное соотношение: \par
	Соотношение \eqref{eq:recurrentCrankNicholson} является \textbf{рекуррентным прогоночным соотношением}.
		
	\subsection*{Задание 16}
	\addcontentsline{toc}{subsection}{Задание 16}
	\large
	Составить алгоритм (блок-схему) расчёта:
	\tikzstyle{start} = [circle, draw=black!60, fill=white!5, very thick, minimum size=13mm]
	\tikzstyle{stop} = [circle, draw=black!60, fill=white!60, very thick, minimum size=13mm]
	\tikzstyle{process} = [rectangle, minimum width=3cm, minimum height=1cm, text centered, draw=black]
	\tikzstyle{decision} = [diamond, minimum width=3cm, minimum height=1cm, text centered, draw=black]
	\tikzstyle{arrow} = [thick,->,>=stealth]
	\begin{center}
		\begin{tikzpicture}[node distance=2cm]
			\node (start) [start] {Start};
			\node (in1) [process, below of=start, yshift=-1cm] {
				\makecell{
					Задание начальных условий: \\
					цикл по $j=1..N$:\\ 
					$u_{j}^{0} = (j-1)h$
				}
			};
			\node (in2) [process, below of=in1] {
				$n = 0$
			};
			\node (pr1) [process, below of=in2, yshift=-0.5cm] {
				\makecell{
					Определение $\alpha_{1}$, $\beta_{1}$ из левого граничного условия \\
					$\alpha_{1} = 0$ \\
					$\beta_{1} = 0$
				}
			};
			\node (pr2) [process, below of=pr1, yshift=-1cm] {
				\makecell{
					цикл по $j=2..N-1$: \\
					расчёт $a_{j}$, $b_{j}$, $c_{j}$, $\xi_{j}^{n}$; $\alpha_{j}=-\frac{\alpha{j}}{b_{j}+c_{j}\alpha_{j-1}}$, $\beta_{j}=\frac{\xi_{j}^{n}-c_{j}\beta{j-1}}{b_{j}+c_{j}\alpha_{j-1}}$
				}
			};
			\node (pr3) [process, below of=pr2, yshift=-0.5cm] {
				\makecell{
					Определение $u_{N}^{n+1}$ из правого граничного условия \\
					$u_{N}^{n+1} = \frac{\beta_{N-1} + h}{1 - \alpha_{N-1}}$
				}
			};
			\node (pr4) [process, below of=pr3, yshift=-0.5cm] {
				\makecell{
					цикл по $j=N-1..1$: \\
					$u_{j}^{n+1} = \alpha_{j}u_{j+1}^{n+1} + \beta_{j}$
				}
			};
			\node (dec) [decision, below of=pr4, yshift=-2cm] {
				$\lvert \lvert u^{n+1} - u^{n} \rvert \rvert \le \epsilon$
			};
			\node (stop) [stop, right of=dec, xshift=2cm] {End};
			\node (pr5) [process, below of=dec, yshift=-2cm] {
				$n = n + 1$
			};
			\draw [arrow] (start) -- (in1);
			\draw [arrow] (in1) -- (in2);
			\draw [arrow] (in2) -- (pr1);
			\draw [arrow] (pr1) -- (pr2);
			\draw [arrow] (pr2) -- (pr3);
			\draw [arrow] (pr3) -- (pr4);
			\draw [arrow] (pr4) -- (dec);
			\draw [arrow] (dec) -- node[anchor=west] {нет} (pr5);
			\draw [arrow] (dec) -- node[anchor=south] {да} (stop);
			\draw [arrow] (pr5.west) -- ++(left:8.2cm) -| +(up:16cm) -- (pr1.west);
		\end{tikzpicture}
	\end{center}

	\subsection*{Задание 17}
	\addcontentsline{toc}{subsection}{Задание 17}
	\large
	Представить задачу в нестационарном виде: \par
	Представлю стационарную задачу в нестационарном виду. Для этого в уравнение необходимо добавить фиктивную производную по времени:
	\begin{equation}\label{eq:nonstationarySecond}
		-\frac{du}{dx}=4\frac{d^{2}u}{dx^{2}}+2x \rightarrow \frac{\partial \tilde{u}}{\partial \tau}-\frac{\partial \tilde{u}}{\partial x}=4\frac{\partial^{2} \tilde{u}}{\partial x^{2}}+2x.
	\end{equation}
	\par
	При этом искомая функция станет уже функцией двух переменных:
	\begin{equation*}
		u(x) \rightarrow \tilde{u}(x,\tau).
	\end{equation*}
	
	\subsection*{Задание 18}
	\addcontentsline{toc}{subsection}{Задание 18}
	\large
	Записать явную разностную схему для уравнения \eqref{eq:nonstationarySecond}:
	\begin{equation}\label{eq:explicitSecond}
		\frac{u_{j}^{n+1} - u_{j}^{n}}{\Delta t} - \frac{u_{j+1}^{n} - u_{j}^{n}}{h} = 4\frac{u_{j+1}^{n} - u_{j}^{n} + u_{j-1}^{n}}{h^{2}} + 2(j-1)h.
	\end{equation}

	\subsection*{Задание 19}
	\addcontentsline{toc}{subsection}{Задание 19}
	\large
	Вывести рекуррентное соотношение для уравнения \eqref{eq:explicitSecond}: \par
	Выражая из разностной схемы \eqref{eq:explicitSecond} величину $u_{j}^{n+1}$, получаю рекуррентное соотношение
	\begin{equation*}
		u_{j}^{n+1} = u_{j}^{n} + \Delta t[\frac{1}{h}(u_{j+1}^{n} - u_{j}^{n}) + \frac{4}{h^{2}}(u_{j+1}^{n} - u_{j}^{n} + u_{j-1}^{n}) + 2(j-1)h],
	\end{equation*}
	которое с учётом равенства $\Delta t = \frac{h^{2}}{h+2}$ преобразуется к виду:
	\begin{equation*}
		\makecell{
			u_{j}^{n+1} = u_{j}^{n} + \frac{h^{2}}{h+8}[(\frac{4}{h^{2}} + \frac{1}{h})u_{j+1}^{n} - (\frac{8}{h^{2}}+\frac{1}{h})u_{j}^{n} + \frac{4}{h^{2}}u_{j-1}^{n} + 2(j-1)h] \Rightarrow \\
			\Rightarrow u_{j}^{n+1} = \frac{u_{j-1}^{n} + (4+h)u_{j+1}^{n} + 2(j-1)h}{h+8}.
		}
	\end{equation*}

	\subsection*{Задание 20}
	\addcontentsline{toc}{subsection}{Задание 20}
	\large
	Составить алгоритм (блок-схему) расчёта:
	\tikzstyle{start} = [circle, draw=black!60, fill=white!5, very thick, minimum size=13mm]
	\tikzstyle{stop} = [circle, draw=black!60, fill=white!60, very thick, minimum size=13mm]
	\tikzstyle{process} = [rectangle, minimum width=3cm, minimum height=1cm, text centered, draw=black]
	\tikzstyle{decision} = [diamond, minimum width=3cm, minimum height=1cm, text centered, draw=black]
	\tikzstyle{arrow} = [thick,->,>=stealth]
	\begin{center}
		\begin{tikzpicture}[node distance=2cm]
			\node (start) [start] {Start};
			\node (in1) [process, below of=start, yshift=-1cm] {
				\makecell{
					Задание начальных условий: \\
					цикл по $j=1..N$:\\ 
					$u_{j}^{0} = (j-1)h$
				}
			};
			\node (in2) [process, below of=in1] {
				$n = 0$
			};
			\node (pr1) [process, below of=in2, yshift=-0.5cm] {
				\makecell{
					Определение $\alpha_{1}$, $\beta_{1}$ из левого граничного условия \\
					$\alpha_{1} = 0$ \\
					$\beta_{1} = 0$
				}
			};
			\node (pr2) [process, below of=pr1, yshift=-1cm] {
				\makecell{
					цикл по $j=2..N-1$: \\
					расчёт $a_{j}$, $b_{j}$, $c_{j}$, $\xi_{j}^{n}$; $\alpha_{j}=-\frac{\alpha{j}}{b_{j}+c_{j}\alpha_{j-1}}$, $\beta_{j}=\frac{\xi_{j}^{n}-c_{j}\beta{j-1}}{b_{j}+c_{j}\alpha_{j-1}}$
				}
			};
			\node (pr3) [process, below of=pr2, yshift=-0.5cm] {
				\makecell{
					Определение $u_{N}^{n+1}$ из правого граничного условия \\
					$u_{N}^{n+1} = \frac{\beta_{N-1} + h}{1 - \alpha_{N-1}}$
				}
			};
			\node (pr4) [process, below of=pr3, yshift=-0.5cm] {
				\makecell{
					цикл по $j=N-1..1$: \\
					$u_{j}^{n+1} = \alpha_{j}u_{j+1}^{n+1} + \beta_{j}$
				}
			};
			\node (dec) [decision, below of=pr4, yshift=-2cm] {
				$\lvert \lvert u^{n+1} - u^{n} \rvert \rvert \le \epsilon$
			};
			\node (stop) [stop, right of=dec, xshift=2cm] {End};
			\node (pr5) [process, below of=dec, yshift=-2cm] {
				$n = n + 1$
			};
			\draw [arrow] (start) -- (in1);
			\draw [arrow] (in1) -- (in2);
			\draw [arrow] (in2) -- (pr1);
			\draw [arrow] (pr1) -- (pr2);
			\draw [arrow] (pr2) -- (pr3);
			\draw [arrow] (pr3) -- (pr4);
			\draw [arrow] (pr4) -- (dec);
			\draw [arrow] (dec) -- node[anchor=west] {нет} (pr5);
			\draw [arrow] (dec) -- node[anchor=south] {да} (stop);
			\draw [arrow] (pr5.west) -- ++(left:8.2cm) -| +(up:16cm) -- (pr1.west);
		\end{tikzpicture}
	\end{center}
\end{document}
