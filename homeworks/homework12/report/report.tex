\documentclass[12pt, a4paper]{report}
\usepackage[top=1cm, left=1cm, right=1cm]{geometry}

\usepackage[utf8]{inputenc}
\usepackage[russian]{babel}

\usepackage{array}
\newcolumntype{M}[1]{>{\centering\arraybackslash}m{#1}}

\usepackage{hyperref}
\hypersetup{
	colorlinks,
	citecolor=black,
	filecolor=black,
	linkcolor=black,
	urlcolor=black
}

\usepackage{sectsty}
\allsectionsfont{\centering}

\usepackage{indentfirst}
\setlength\parindent{24pt}

\usepackage{makecell}

\usepackage{amsmath}

\def\H{\rule{0pt}{1.5ex}H}

\usepackage{tikz}
\usetikzlibrary{shapes.geometric, arrows}
\usetikzlibrary{arrows.meta}
\usetikzlibrary{decorations.text}

\begin{document}
	\begin{titlepage}
		\begin{center}
			\large \textbf{Министерство науки и высшего образования Российской Федерации} \\
			\large \textbf{Федеральное государственное бюджетное образовательное учреждение высшего образования} \\
			\large \textbf{«Российский химико-технологический университет имени Д.И. Менделеева»} \\

			\vspace*{4cm}
			\LARGE \textbf{ОТЧЕТ ПО ДОМАШНЕЙ РАБОТЕ №12}

			\vspace*{4cm}
			\begin{flushright}
				\Large
				\begin{tabular}{>{\raggedleft\arraybackslash}p{9cm} p{10cm}}
					Выполнил студент группы КС-36: & Золотухин Андрей Александрович \\
					Ссылка на репозиторий: & https://github.com/ \\
					& CorgiPuppy/ \\
					& num-methods-eq-math-phys-chem-labs \\
					Приняла: & Кольцова Элеонора Моисеевна \\
					Дата сдачи: & 12.05.2025 \\
				\end{tabular}
			\end{flushright}

			\vspace*{6cm}
			\Large \textbf{Москва \\ 2025}
		\end{center}
	\end{titlepage}

	\tableofcontents
	\thispagestyle{empty}
	\newpage

	\pagenumbering{arabic}

	\section*{Описание задачи}
	\addcontentsline{toc}{section}{Описание задачи}
	\large
	\begin{center}
		\begin{tabular}{||c|c|c|c||}
			\hline
			Уравнение & Интервал переменной & Метод & Граничные условия \\

			\hline
			$ \frac{du}{dx}=\frac{d^{2}u}{dx^{2}}+x$ & $x \in [0, 1]$ & \makecell{Установление с \\ явной схемой; \\ Установление с \\ неявной схемой; \\ Установление со \\ схемой Кранка- \\ Николсона} & \makecell{$ \begin{cases} u(t, x = 0) = 0 \\ \frac{du}{dx}(t, x = 1) = 1 \end{cases}$} \\

			\hline
		\end{tabular}
	\end{center}
	\par
	Для заданного уравнения:
	\begin{enumerate}
		\item представить задачу в нестационарном виде;
		\item записать явную разностную схему;
		\item вывести рекуррентное соотношение;
		\item составить алгоритм (блок-схему) расчёта;
		\item записать неявную разностную схему;
		\item привести схему к виду, удобному для использования метода прогонки;
		\item проверить сходимость прогонки;
		\item найти $\alpha_1$, $\beta_1$, $u_{N}^{n+1}$;
		\item записать рекуррентное прогоночное соотношение;
		\item составить алгоритм (блок-схему) расчёта;
		\item записать разностную схему Кранка-Николсона;
		\item привести схему к виду, удобному для использования метода прогонки;
		\item проверить сходимость прогонки;
		\item найти $\alpha_1$, $\beta_1$, $u_{N}^{n+1}$;
		\item записать рекуррентное прогоночное соотношение;
		\item составить алгоритм (блок-схему) расчёта;
	\end{enumerate}

	\begin{center}
		\begin{tabular}{||c|c|c|c||}
			\hline
			Уравнение & Интервал переменной & Метод & Граничные условия \\

			\hline
			$ -\frac{du}{dx}=4\frac{d^{2}u}{dx^{2}}+2x$ & $x \in [0, 1]$ & \makecell{Установление с \\ явной схемой; \\ Установление со \\ схемой Кранка- \\ Николсона} & \makecell{$ \begin{cases} \frac{du}{dx}(t, x = 0) = 2 \\ u(t, x = 1) = 2 \end{cases}$} \\

			\hline
		\end{tabular}
	\end{center}
	\par
	Для заданного уравнения:
	\begin{enumerate}
		\setcounter{enumi}{16}
		\item представить задачу в нестационарном виде;
		\item записать явную разностную схему;
		\item вывести рекуррентное соотношение;
		\item составить алгоритм (блок-схему) расчёта;
		\item записать неявную разностную схему;
		\item привести схему к виду, удобному для использования метода прогонки;
		\item проверить сходимость прогонки;
		\item найти $\alpha_1$, $\beta_1$, $u_{N}^{n+1}$;
		\item записать рекуррентное прогоночное соотношение;
		\item составить алгоритм (блок-схему) расчёта;
		\item записать разностную схему Кранка-Николсона;
		\item привести схему к виду, удобному для использования метода прогонки;
		\item проверить сходимость прогонки;
		\item найти $\alpha_1$, $\beta_1$, $u_{N}^{n+1}$;
		\item записать рекуррентное прогоночное соотношение;
		\item составить алгоритм (блок-схему) расчёта;
	\end{enumerate}

	\newpage

	\section*{Выполнение задачи}
	\addcontentsline{toc}{section}{Выполнение задачи}

	\subsection*{Задание 1}
	\addcontentsline{toc}{subsection}{Задание 1}
	\large
	Представить задачу в нестационарном виде: \par
	Представлю стационарную задачу в нестационарном виду. Для этого в уравнение необходимо добавить фиктивную производную по времени:
	\begin{equation}\label{eq:nonstationary}
		\frac{du}{dx}=\frac{d^{2}u}{dx^{2}}+x \rightarrow \frac{\partial \tilde{u}}{\partial \tau}+\frac{\partial \tilde{u}}{\partial x}=\frac{\partial^{2} \tilde{u}}{\partial x^{2}}+x.
	\end{equation}
	\par
	При этом искомая функция станет уже функцией двух переменных:
	\begin{equation*}
		u(x) \rightarrow \tilde{u}(x,\tau).
	\end{equation*}
	
	\subsection*{Задание 2}
	\addcontentsline{toc}{subsection}{Задание 2}
	\large
	Записать явную разностную схему для уравнения \eqref{eq:nonstationary}:
	\normalsize
	\begin{equation}\label{eq:explicit}
		\frac{u_{j}^{n+1} - u_{j}^{n}}{\Delta t} + \frac{u_{j}^{n} - u_{j-1}^{n}}{h} = \frac{u_{j+1}^{n} - u_{j}^{n} + u_{j-1}^{n}}{h^{2}} + (i-1)h.
	\end{equation}

	\subsection*{Задание 3}
	\addcontentsline{toc}{subsection}{Задание 3}
	\large
	Вывести рекуррентное соотношение для уравнения \eqref{eq:explicit}: \par
	Выражая из разностной схемы \eqref{eq:explicit} величину $u_{j}^{n+1}$, получаю рекуррентное соотношение
	\begin{equation*}
		u_{j}^{n+1} = u_{j}^{n} + \Delta t[\frac{1}{h}
	\end{equation*}

\end{document}
