\documentclass[12pt, a4paper]{report}
\usepackage[top=1cm, left=1cm, right=1cm]{geometry}

\usepackage[utf8]{inputenc}
\usepackage[russian]{babel}

\usepackage{array}
\newcolumntype{M}[1]{>{\centering\arraybackslash}m{#1}}

\usepackage{hyperref}
\hypersetup{
	colorlinks,
	citecolor=black,
	filecolor=black,
	linkcolor=black,
	urlcolor=black
}

\usepackage{sectsty}
\allsectionsfont{\centering}

\usepackage{indentfirst}
\setlength\parindent{24pt}

\usepackage{makecell}

\usepackage{amsmath}

\def\H{\rule{0pt}{1.5ex}H}

\usepackage{tikz}
\usetikzlibrary{shapes.geometric, arrows}
\usetikzlibrary{arrows.meta}
\usetikzlibrary{decorations.text}

\begin{document}
	\begin{titlepage}
		\begin{center}
			\large \textbf{Министерство науки и высшего образования Российской Федерации} \\
			\large \textbf{Федеральное государственное бюджетное образовательное учреждение высшего образования} \\
			\large \textbf{«Российский химико-технологический университет имени Д.И. Менделеева»} \\

			\vspace*{4cm}
			\LARGE \textbf{ОТЧЕТ ПО ДОМАШНЕЙ РАБОТЕ №13} \\
			\Large \textbf{5 ВАРИАНТ}

			\vspace*{3cm}
			\begin{flushright}
				\Large
				\begin{tabular}{>{\raggedleft\arraybackslash}p{9cm} p{10cm}}
					Выполнил студент группы КС-36: & Золотухин Андрей Александрович \\
					Ссылка на репозиторий: & https://github.com/ \\
					& CorgiPuppy/ \\
					& num-methods-eq-math-phys-chem-labs \\
					Приняла: & Кольцова Элеонора Моисеевна \\
					Дата сдачи: & 19.05.2025 \\
				\end{tabular}
			\end{flushright}

			\vspace*{6cm}
			\Large \textbf{Москва \\ 2025}
		\end{center}
	\end{titlepage}

	\tableofcontents
	\thispagestyle{empty}
	\newpage

	\pagenumbering{arabic}

	\section*{Описание задачи}
	\addcontentsline{toc}{section}{Описание задачи}
	\large
	\begin{center}
		\begin{tabular}{||c|c|c||}
			\hline
			Уравнение & \makecell{Интервалы \\ переменных} & Начальные и граничные условия \\

			\hline
			\small 
			$ \frac{\partial u}{\partial t} - \frac{\partial u}{\partial x} - \frac{\partial u}{\partial y} = y\frac{\partial^{2} u}{\partial x^{2}} + x\frac{\partial^{2} u}{\partial y^{2}} - 3u^{2} $ & \makecell{$ x \in [0, 1] $ \\ $ y \in [0, 1] $ \\ $ t \in [0, 1] $} & \makecell{$ u(t = 0, x, y) = 0 $ \\ $\begin{cases} u(t, x = 0, y) = ty \\ u(t, x = 1, y) = 0 \end{cases}$ \\ $\begin{cases} u(t, x, y = 0) = tx \\ u(t, x, y = 1) = 2 \end{cases}$} \\

			\hline
		\end{tabular}
	\end{center}

	Для заданного уравнения:
	\begin{enumerate}
		\item записать неявную разностную схему;
		\item записать схему расщепления;
		\item привести схемы к виду, удобному для использования метода прогонки;
		\item проверить сходимость прогонки;
		\item записать рекуррентное прогоночное соотношение;
		\item составить алгоритм (блок-схему) расчёта.
	\end{enumerate}

	\begin{center}
		\begin{tabular}{||c|c|c||}
			\hline
			Уравнение & \makecell{Интервалы \\ переменных} & Начальные и граничные условия \\

			\hline
			$ \frac{\partial u}{\partial t} - \frac{\partial u}{\partial x} = \frac{\partial u}{\partial y} - e^{txy} $ & \makecell{$ x \in [0, 1] $ \\ $ y \in [0, 1] $ \\ $ t \in [0, 1] $} & \makecell{$ u(t = 0, x, y) = 1 $ \\ $ \begin{cases} u(t, x = 0, y) = 1 \\ u(t, x = 1, y) = e^{y} \end{cases}$ \\ $ \begin{cases} u(t, x, y = 0) = 1 \\ u(t, x, y = 1) = e^{x} \end{cases} $} \\

			\hline
		\end{tabular}
	\end{center}
	\par
	Для заданного уравнения:
	\begin{enumerate}
		\setcounter{enumi}{6}
		\item записать неявную разностную схему;
		\item записать схему переменных направлений;
		\item записать рекуррентное соотношение;
		\item составить алгоритм (блок-схему) расчёта.
	\end{enumerate}
	
	\begin{center}
		\begin{tabular}{||c|c|c||}
			\hline
			Уравнение & \makecell{Интервалы \\ переменных} & Начальные и граничные условия \\

			\hline
			\small 
			$ -\frac{du}{dx} + 2x\frac{d^{2}u}{dx^{2}} = 5u $ & $ x \in [0, 1] $ & $\begin{cases} \frac{du}{dx}(x = 0) = u(x = 0) \\ \frac{du}{dx}(x = 1) = 2u(x = 1) \end{cases}$ \\

			\hline
		\end{tabular}
	\end{center}
	\par
	Для заданного уравнения:
	\begin{enumerate}
		\setcounter{enumi}{10}
		\item представить задачу в нестационарном виде;
		\item записать неявную разностную схему;
		\item привести схемы к виду, удобному для использования метода прогонки;
		\item проверить сходимость прогонки;
		\item записать итерационное прогоночное соотношение;
		\item записать условие для окончания итерационного процесса;
		\item записать начальное приближение;
		\item составить алгоритм (блок-схему) расчёта;
	\end{enumerate}

	\begin{center}
		\begin{tabular}{||c|c|c||}
			\hline
			Уравнение & \makecell{Интервалы \\ переменных} & Начальные и граничные условия \\

			\hline
			\small 
			$ \frac{\partial u}{\partial t} + \frac{\partial u}{\partial x} - 2\frac{\partial u}{\partial y}) = 7t(\frac{\partial^{2} u}{\partial x^{2}} + \frac{\partial^{2} u}{\partial x^{2}}) + t^{2} $ & \makecell{$ x \in [0, 1] $ \\ $ y \in [0, 1] $ \\ $ t \in [0, 1] $} & \makecell{$ u(t = 0, x, y) = y $ \\ $\begin{cases} u(t, x = 0, y) = 0 \\ u(t, x = 1, y) = t \end{cases}$ \\ $\begin{cases} u(t, x, y = 0) = 0 \\ u(t, x, y = 1) = x \end{cases}$} \\

			\hline
		\end{tabular}
	\end{center}
	\par
	Для заданного уравнения:
	\begin{enumerate}
		\setcounter{enumi}{18}
		\item записать схему предиктор-корректор;
		\item записать рекуррентное прогоночное соотношение для предиктора;
		\item записать рекуррентное прогоночное соотношение для корректора;
		\item указать порядок аппроксимации разностной схемы;
	\end{enumerate}

	\newpage

	\section*{Выполнение задачи}
	\addcontentsline{toc}{section}{Выполнение задачи}

	\subsection*{Задание 1}
	\addcontentsline{toc}{subsection}{Задание 1}
	\large

	\subsection*{Задание 2}
	\addcontentsline{toc}{subsection}{Задание 2}
	\large

	\subsection*{Задание 3}
	\addcontentsline{toc}{subsection}{Задание 3}
	\large

	\subsection*{Задание 4}
	\addcontentsline{toc}{subsection}{Задание 4}
	\large

	\subsection*{Задание 5}
	\addcontentsline{toc}{subsection}{Задание 5}
	\large

	\subsection*{Задание 6}
	\addcontentsline{toc}{subsection}{Задание 6}
	\large

	\subsection*{Задание 7}
	\addcontentsline{toc}{subsection}{Задание 7}
	\large

	\subsection*{Задание 8}
	\addcontentsline{toc}{subsection}{Задание 8}
	\large

	\subsection*{Задание 9}
	\addcontentsline{toc}{subsection}{Задание 9}
	\large

	\subsection*{Задание 10}
	\addcontentsline{toc}{subsection}{Задание 10}
	\large

	\subsection*{Задание 11}
	\addcontentsline{toc}{subsection}{Задание 11}
	\large

	\subsection*{Задание 12}
	\addcontentsline{toc}{subsection}{Задание 12}
	\large

	\subsection*{Задание 13}
	\addcontentsline{toc}{subsection}{Задание 13}
	\large

	\subsection*{Задание 14}
	\addcontentsline{toc}{subsection}{Задание 14}
	\large

	\subsection*{Задание 15}
	\addcontentsline{toc}{subsection}{Задание 15}
	\large

	\subsection*{Задание 16}
	\addcontentsline{toc}{subsection}{Задание 16}
	\large

	\subsection*{Задание 17}
	\addcontentsline{toc}{subsection}{Задание 17}
	\large

	\subsection*{Задание 18}
	\addcontentsline{toc}{subsection}{Задание 18}
	\large

	\subsection*{Задание 19}
	\addcontentsline{toc}{subsection}{Задание 19}
	\large

	\subsection*{Задание 20}
	\addcontentsline{toc}{subsection}{Задание 20}
	\large

	\subsection*{Задание 21}
	\addcontentsline{toc}{subsection}{Задание 21}
	\large

	\subsection*{Задание 22}
	\addcontentsline{toc}{subsection}{Задание 22}
	\large
\end{document}
