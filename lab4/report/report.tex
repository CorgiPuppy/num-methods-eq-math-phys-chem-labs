\documentclass[12pt, a4paper]{report}
\usepackage[top=1cm, left=1cm, right=1cm]{geometry}

\usepackage[utf8]{inputenc}
\usepackage[russian]{babel}

\usepackage{array}
\newcolumntype{M}[1]{>{\centering\arraybackslash}m{#1}}

\usepackage{hyperref}
\hypersetup{
	colorlinks,
	citecolor=black,
	filecolor=black,
	linkcolor=black,
	urlcolor=black
}

\usepackage{sectsty}
\allsectionsfont{\centering}

\usepackage{indentfirst}
\setlength\parindent{24pt}

\usepackage{makecell}

\usepackage{amsmath}

\usepackage{tikz}
\usetikzlibrary{shapes.geometric, arrows}
\usetikzlibrary{arrows.meta}
\usetikzlibrary{decorations.text}

\usepackage{listings}
\usepackage{xcolor}
\definecolor{codegreen}{rgb}{0,0.6,0}
\definecolor{codegray}{rgb}{0.5,0.5,0.5}
\definecolor{codepurple}{rgb}{0.58,0,0.82}
\definecolor{backcolour}{rgb}{0.95,0.95,0.92}
\lstdefinestyle{mystyle}{
    backgroundcolor=\color{backcolour},
    commentstyle=\color{codegreen},
    keywordstyle=\color{magenta},
    numberstyle=\normalsize\color{codegray},
    stringstyle=\color{codepurple},
    basicstyle=\ttfamily\footnotesize,
    breakatwhitespace=false,
    breaklines=true,
    captionpos=b,
    keepspaces=true,
    numbers=left,
    numbersep=5pt,
    showspaces=false,
    showstringspaces=false,
    showtabs=false,
    tabsize=2
}

\usepackage{graphicx}
\graphicspath{ {plots/pictures/} }

\usepackage{booktabs}
\usepackage{csvsimple}
\usepackage{longtable}
\usepackage{caption}

\begin{document}
	\begin{titlepage}
		\begin{center}
			\large \textbf{Министерство науки и высшего образования Российской Федерации} \\
			\large \textbf{Федеральное государственное бюджетное образовательное учреждение высшего образования} \\
			\large \textbf{«Российский химико-технологический университет имени Д.И. Менделеева»} \\

			\vspace*{4cm}
			\LARGE \textbf{ОТЧЕТ ПО ЛАБОРАТОРНОЙ РАБОТЕ №4}

			\vspace*{1cm}
			\LARGE \textbf{Вариант 9}

			\vspace*{4cm}
			\begin{flushright}
				\Large
				\begin{tabular}{>{\raggedleft\arraybackslash}p{9cm} p{10cm}}
					Выполнил студент группы КС-36: & Золотухин Андрей Александрович \\
					Ссылка на репозиторий: & https://github.com/ \\
					& CorgiPuppy/ \\
					& num-methods-eq-math-phys-chem-labs \\
					Принял: & Лебедев Данила Александрович \\
					Дата сдачи: & 07.05.2025 \\
				\end{tabular}
			\end{flushright}

			\vspace*{4cm}
			\Large \textbf{Москва \\ 2025}
		\end{center}
	\end{titlepage}

	\tableofcontents
	\thispagestyle{empty}
	\newpage

	\pagenumbering{arabic}

	\section*{Описание задачи}
	\addcontentsline{toc}{section}{Описание задачи}
	\large
	\begin{center}
		\begin{tabular}{||c|c|c|c||}
			\hline
			Вариант & Уравнение & Метод & Граничные условия \\
			\hline
			9 & $ \frac{du}{dx}=\frac{d^{2}u}{dx^{2}}+4$ & \makecell{Установление со \\ схемой Кранка- \\ Николсона} & \makecell{$ \begin{cases} u(x = 0) = 1 \\ u(x = 1) = 6.7 \end{cases}$} \\

			\hline
		\end{tabular}
	\end{center}
	\par
	Для заданного уравнения:
	\begin{enumerate}
		\item представить задачу в нестационарном виде;
		\item записать разностную схему Кранка-Николсона;
		\item привести схему к виду, удобному для использования метода прогонки;
		\item проверить сходимость прогонки;
		\item найти $\alpha_1$, $\beta_1$, $u_{N}^{n+1}$;
		\item записать рекуррентное прогоночное соотношение;
		\item составить алгоритм (блок-схему) расчёта;
		\item построить программу на любом удобном языке программирования;
		\item построить численный расчёт с использованием различных значений $\Delta t$ \textit{= \{0.1;} \textit{0.01;} \textit{0.001\}}, \textit{h = \{0.1; 0.01\}};
		\item Сравнить результаты вычислений между собой в точках: \textit{x = \{0;} \textit{0.1;} \textit{0.2;} \textit{0.3;} \textit{0.4;} \textit{0.5;} \textit{0.6;} \textit{0.7;} \textit{0.8;} \textit{0.9;} \textit{1\}}.
	\end{enumerate}

	\newpage

	\section*{Выполнение задачи}
	\addcontentsline{toc}{section}{Выполнение задачи}

	\subsection*{Задание 1}
	\addcontentsline{toc}{subsection}{Задание 1}
	\large
	Представить задачу в нестационарном виде: \par
	Представлю стационарную задачу в нестационарном виду. Для этого в уравнение необходимо добавить фиктивную производную по времени:
	\begin{equation}\label{eq:nonstationary}
		\frac{du}{dx}=\frac{d^{2}u}{dx^{2}}+4 \rightarrow \frac{\partial \tilde{u}}{\partial \tau}+\frac{\partial \tilde{u}}{\partial x}=\frac{\partial^{2} \tilde{u}}{\partial x^{2}}+4.
	\end{equation}
	\par
	При этом искомая функция станет уже функцией двух переменных:
	\begin{equation*}
		u(x) \rightarrow \tilde{u}(x,\tau).
	\end{equation*}
	
	\subsection*{Задание 2}
	\addcontentsline{toc}{subsection}{Задание 2}
	\large
	Записать разностную схему Кранка-Николсона для уравнения \eqref{eq:nonstationary}:
	\normalsize
	\begin{equation}\label{eq:crankNicholson}
		\frac{u_{j}^{n+1} - u_{j}^{n}}{\Delta t} + \frac{1}{2}\frac{u_{j}^{n+1} - u_{j-1}^{n+1}}{h} + \frac{1}{2}\frac{u_{j}^{n} - u_{j-1}^{n}}{h} = \frac{1}{2}\frac{u_{j+1}^{n+1} - u_{j}^{n+1} + u_{j-1}^{n+1}}{h^{2}} + \frac{1}{2}\frac{u_{j+1}^{n} - u_{j}^{n} + u_{j-1}^{n}}{h^{2}} + 4.
	\end{equation}

	\subsection*{Задание 3}
	\addcontentsline{toc}{subsection}{Задание 3}
	\large
	Привести схему к виду, удобному для использования метода прогонки:
	\small
	\begin{equation*}
		-\frac{1}{2}\frac{\Delta t}{h^{2}}u_{j+1}^{n+1} + (1+\frac{1}{2}\frac{\Delta t}{h} + \frac{\Delta t}{h^{2}})u_{j}^{n+1} - (\frac{1}{2}\frac{\Delta t}{h} + \frac{1}{2}\frac{\Delta t}{h^{2}})u_{j-1}^{n+1} = u_{j}^{n} + \frac{1}{2}\frac{\Delta t}{h^{2}}(u_{j+1}^{n}-2u_{j}^{n}+u_{j-1}^{n}) - \frac{1}{2}\frac{\Delta t}{h}(u_{j}^{n}-u_{j-1}^{n}) + \Delta t 4.
	\end{equation*}
	\large
	\par
	Введу следующие обозначения:
	\begin{equation*}
		\makecell{
			a_{j}=-\frac{1}{2}\frac{\Delta t}{h^{2}}, \> b_{j}=1+\frac{1}{2}\frac{\Delta t}{h} + \frac{\Delta t}{h^{2}}, \> c_{j}=-(\frac{1}{2}\frac{\Delta t}{h} + \frac{1}{2}\frac{\Delta t}{h^{2}}), \\
			\xi_{j}^{n}=u_{j}^{n} + \frac{1}{2}\frac{\Delta t}{h^{2}}(u_{j+1}^{n}-2u_{j}^{n}+u_{j-1}^{n}) - \frac{1}{2}\frac{\Delta t}{h}(u_{j}^{n}-u_{j-1}^{n}) + \Delta t 4.
		}
	\end{equation*}
	\par
	С учётом обозначений равенство будет иметь вид:
	\begin{equation*}
		\alpha_{j}u_{j+1}^{n+1} + b_{j}u_{j}^{n+1} + c_{j}u_{j-1}^{n+1} = \xi_{j}^{n}.
	\end{equation*}
	\par
	Данное преобразование называется \textit{преобразованием} \textit{схемы Кранка-Николсона} \textit{к виду}, \textit{удобному для} \textit{использования метода прогонки}.

	\subsection*{Задание 4}
	\addcontentsline{toc}{subsection}{Задание 4}
	\large
	Проверить сходимость прогонки: \par
	Легко видеть, что для разностной схемы \eqref{eq:crankNicholson} достаточное условие сходимости прогонки выполняется:
	\begin{equation*}
		\lvert a_{j} \rvert + \lvert c_{j} \rvert = \frac{1}{2}\frac{\Delta t}{h} + \frac{\Delta t}{h^{2}} < 1 + \frac{1}{2}\frac{\Delta t}{h} + \frac{\Delta t}{h^{2}} = \lvert b_{j} \rvert.
	\end{equation*}

	\subsection*{Задание 5}
	\addcontentsline{toc}{subsection}{Задание 5}
	\large
	Найти $\alpha_1$, $\beta_1$, $u_{N}^{n+1}$: \par
	Для реализации разностной схемы Кранка-Николсона требуется ввести некоторое дополнительное условие, связывающее значения функции \textit{u(t, x)} на \textit{(n+1)}-м шаге по времени. Представлю это дополнительное условие в виде линейной зависимости
	\begin{equation}\label{eq:recurrent}
		u_{j}^{n+1} = \alpha_{j} u_{j+1}^{n+1} + \beta{j},
	\end{equation}
	справедливой для любого из значений $j=1..N-1$. \par
	Соотношение \eqref{eq:recurrent} называют \textbf{рекуррентным прогоночным соотношением}, а коэффициенты $\alpha_{j}$, $\beta_{j}$ - \textbf{прогоночными коэффициентами}. \\
	\par
	Для определения прогоночных коэффициентов на \textit{1}-м шаге по координате \textit{x}, использую рекуррентное прогоночное соотношение \eqref{eq:recurrent}, записанное для $j=1$:
	\begin{equation*}
		u_{1}^{n+1} = \alpha_{1} u_{2}^{n+1} + \beta_{1}
	\end{equation*}
	и левое граничное условие:
	\begin{equation*}
		u_{1}^{n+1} = 1.
	\end{equation*}
	\par
	Сравнивая эти два соотношения, получаю:
	\begin{equation*}
		\alpha_{1} = 0, \beta_{1} = 1.
	\end{equation*}
	\par
	Значение функции \textit{u(t, x)} на \textit{(n+1)}-м шаге по времени в крайней правой точке, которое можно определить из правого граничного условия:
	\begin{equation*}
		u_{N}^{n+1} = 6.7.
	\end{equation*}
	
	\subsection*{Задание 6}
	\addcontentsline{toc}{subsection}{Задание 6}
	\large
	Записать рекуррентное прогоночное соотношение: \par
	Соотношение \eqref{eq:recurrent} является \textbf{рекуррентным прогоночным соотношением}.
	
	\subsection*{Задание 7}
	\addcontentsline{toc}{subsection}{Задание 7}
	\large
	Составить алгоритм (блок-схему) расчёта:
	\tikzstyle{start} = [circle, draw=black!60, fill=white!5, very thick, minimum size=13mm]
	\tikzstyle{stop} = [circle, draw=black!60, fill=white!60, very thick, minimum size=13mm]
	\tikzstyle{process} = [rectangle, minimum width=3cm, minimum height=1cm, text centered, draw=black]
	\tikzstyle{decision} = [diamond, minimum width=3cm, minimum height=1cm, text centered, draw=black]
	\tikzstyle{arrow} = [thick,->,>=stealth]
	\begin{center}
		\begin{tikzpicture}[node distance=2cm]
			\node (start) [start] {Start};
			\node (in1) [process, below of=start, yshift=-1cm] {
				\makecell{
					Задание начальных условий: \\
					цикл по $j=1..N$:\\ 
					$u_{j}^{0} = 4$
				}
			};
			\node (in2) [process, below of=in1] {
				$n = 0$
			};
			\node (pr1) [process, below of=in2, yshift=-0.5cm] {
				\makecell{
					Определение $\alpha_{1}$, $\beta_{1}$ из левого граничного условия \\
					$\alpha_{1} = 0$ \\
					$\beta_{1} = 4$
				}
			};
			\node (pr2) [process, below of=pr1, yshift=-1cm] {
				\makecell{
					цикл по $j=2..N-1$: \\
					расчёт $a_{j}$, $b_{j}$, $c_{j}$, $\xi_{j}^{n}$; $\alpha_{j}=-\frac{\alpha{j}}{b_{j}+c_{j}\alpha_{j-1}}$, $\beta_{j}=\frac{\xi_{j}^{n}-c_{j}\beta{j-1}}{b_{j}+c_{j}\alpha_{j-1}}$
				}
			};
			\node (pr3) [process, below of=pr2, yshift=-0.5cm] {
				\makecell{
					Определение $u_{N}^{n+1}$ из правого граничного условия \\
					$u_{N}^{n+1} = 6.7$
				}
			};
			\node (pr4) [process, below of=pr3, yshift=-0.5cm] {
				\makecell{
					цикл по $j=N-1..1$: \\
					$u_{j}^{n+1} = \alpha_{j}u_{j+1}^{n+1} + \beta_{j}$
				}
			};
			\node (dec) [decision, below of=pr4, yshift=-2cm] {
				$\lvert \lvert u^{n+1} - u^{n} \rvert \rvert \le \epsilon$
			};
			\node (stop) [stop, right of=dec, xshift=2cm] {End};
			\node (pr5) [process, below of=dec, yshift=-2cm] {
				$n = n + 1$
			};
			\draw [arrow] (start) -- (in1);
			\draw [arrow] (in1) -- (in2);
			\draw [arrow] (in2) -- (pr1);
			\draw [arrow] (pr1) -- (pr2);
			\draw [arrow] (pr2) -- (pr3);
			\draw [arrow] (pr3) -- (pr4);
			\draw [arrow] (pr4) -- (dec);
			\draw [arrow] (dec) -- node[anchor=west] {нет} (pr5);
			\draw [arrow] (dec) -- node[anchor=south] {да} (stop);
			\draw [arrow] (pr5.west) -- ++(left:8.2cm) -| +(up:16cm) -- (pr1.west);
		\end{tikzpicture}
	\end{center}
	
	\subsection*{Задание 8}
	\addcontentsline{toc}{subsection}{Задание 8}
	\large
	Построить программу на любом удобном языке программирования:
	\lstset{style=mystyle}
	\lstinputlisting[language=Java]{src/main/java/com/stateEquation/Main.java}
\end{document}
