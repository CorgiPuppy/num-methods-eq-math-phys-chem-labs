\documentclass[12pt, a4paper]{report}
\usepackage[top=1cm, left=1cm, right=1cm]{geometry}

\usepackage[utf8]{inputenc}
\usepackage[russian]{babel}

\usepackage{array}
\newcolumntype{M}[1]{>{\centering\arraybackslash}m{#1}}

\usepackage{hyperref}
\hypersetup{
	colorlinks,
	citecolor=black,
	filecolor=black,
	linkcolor=black,
	urlcolor=black
}

\usepackage{sectsty}
\allsectionsfont{\centering}

\usepackage{indentfirst}
\setlength\parindent{24pt}

\usepackage{makecell}

\usepackage{amsmath}

\usepackage{tikz}
\usetikzlibrary{shapes.geometric, arrows}

\usepackage{listings}
\usepackage{xcolor}
\definecolor{codegreen}{rgb}{0,0.6,0}
\definecolor{codegray}{rgb}{0.5,0.5,0.5}
\definecolor{codepurple}{rgb}{0.58,0,0.82}
\definecolor{backcolour}{rgb}{0.95,0.95,0.92}
\lstdefinestyle{mystyle}{
    backgroundcolor=\color{backcolour},
    commentstyle=\color{codegreen},
    keywordstyle=\color{magenta},
    numberstyle=\normalsize\color{codegray},
    stringstyle=\color{codepurple},
    basicstyle=\ttfamily\footnotesize,
    breakatwhitespace=false,
    breaklines=true,
    captionpos=b,
    keepspaces=true,
    numbers=left,
    numbersep=5pt,
    showspaces=false,
    showstringspaces=false,
    showtabs=false,
    tabsize=2
}

\usepackage{graphicx}
\graphicspath{ {plots/pictures/} }

\usepackage{booktabs}
\usepackage{csvsimple}
\usepackage{longtable}
\usepackage{caption}

\begin{document}
	\begin{titlepage}
		\begin{center}
			\large \textbf{Министерство науки и высшего образования Российской Федерации} \\
			\large \textbf{Федеральное государственное бюджетное образовательное учреждение высшего образования} \\
			\large \textbf{«Российский химико-технологический университет имени Д.И. Менделеева»} \\

			\vspace*{4cm}
			\LARGE \textbf{ОТЧЕТ ПО ЛАБОРАТОРНОЙ РАБОТЕ №3}

			\vspace*{1cm}
			\LARGE \textbf{Вариант 22}

			\vspace*{4cm}
			\begin{flushright}
				\Large
				\begin{tabular}{>{\raggedleft\arraybackslash}p{9cm} p{10cm}}
					Выполнил студент группы КС-36: & Золотухин А.А. \\
					Ссылка на репозиторий: & https://github.com/ \\
					& CorgiPuppy/ \\
					& num-methods-eq-math-phys-chem-labs \\
					Принял: & Лебедев Данила Александрович \\
					Дата сдачи: & 11.04.2025 \\
				\end{tabular}
			\end{flushright}

			\vspace*{4cm}
			\Large \textbf{Москва \\ 2025}
		\end{center}
	\end{titlepage}

	\tableofcontents
	\thispagestyle{empty}
	\newpage

	\pagenumbering{arabic}

	\section*{Описание задачи}
	\addcontentsline{toc}{section}{Описание задачи}
	\large
	\begin{center}
		\begin{tabular}{||c|c|c|c||}
			\hline
			Вариант & Уравнение & Интервалы переменных & Начальные и граничные условия \\
			\hline
			22 & $ \frac{\partial u}{\partial t}-2\frac{\partial u}{\partial x}=x$ & \makecell{$ x \in [0, 1] $ \\ $ t \in [0, 1] $} & \makecell{$ u(t = 0, x) = 0 $ \\ $ u(t, x = 0) = t^2 $ \\ $ u(t, x = 1) = t^2+t $} \\

			\hline
		\end{tabular}
	\end{center}
	\par
	Для заданного уравнения:
	\begin{enumerate}
		\item записать неявную разностную схему;
		\item определить порядок аппроксимации разностной схемы;
		\item доказать абсолютную устойчивость разностной схемы (с помощью метода гармоник);
		\item вывести рекуррентное соотношение;
		\item выбор граничного условия зависит от того, с какой конечной разностью вы будете работать (левой или правой). Выбор конечной разности зависит от устойчивости системы. Вы должны выбрать ту конечную разность, при которой схема будет устойчива;
		\item составить алгоритм (блок-схему) расчёта;
		\item построить программу на любом удобном языке программирования;
		\item провести численный расчёт с использованием различных значений $ \Delta t = 0.1 $, $ h = 0.1 $;
		\item сравнить результаты расчётов с истинными значениями функции \textit{u} в соответствующих точках разностной сетки (\textit{истинное} \textit{решение} \textit{уравнения} \textit{будет} \textit{выдано} \textit{преподавателем} \textit{после} \textit{выполнения} \textit{расчётов} \textit{по} \textit{разностной} \textit{схеме});
		\item в случае существенного расхождения результатов расчётов по разностной схеме и истинных значений функции \textit{u} в соответствующих точках разностной сетки выполнить расчёт с меньшими значениями $ \Delta t $ и/или $ h $ (\textit{выбор} \textit{осуществить} \textit{самостоятельно}) с целью получения более точных результатов.
	\end{enumerate}
	\newpage

	\section*{Выполнение задачи}
	\addcontentsline{toc}{section}{Выполнение задачи}

	\subsection*{Задание 1}
	\addcontentsline{toc}{subsection}{Задание 1}
	\large
	Записать неявную разностную схему:
	\begin{equation}\label{eq:implicit}
		\frac{u_{j}^{n+1}-u_{j}^{n}}{\Delta t}-2\frac{u_{j+1}^{n+1}-u_{j}^{n+1}}{h}=(j-1)h.
	\end{equation}
	\par
	В записанной разностной схеме \eqref{eq:implicit} аппроксимация производной функции \textit{u(t, x)} по координате рассматривается на \textit{n+1}-м шаге по времени. Такая разностная схема называется \textbf{неявной}.

	\subsection*{Задание 2}
	\addcontentsline{toc}{subsection}{Задание 2}
	\large
	Определить порядок аппроксимации разностной схемы \eqref{eq:implicit}: \par
\end{document}
